\section{Metodologias e Práticas}
\subsection{Métodos Base}
\label{sec:metodo_base}

Segundo~\citeonline{gutierrez:2009} o método \textit{Scrum} representa um trabalho
em equipe no qual todos os integrantes e envolvidos trabalham para alcançar
o mesmo objetivo, alinhando as mudanças e compartilhando os problemas para que
todos caminhem para a mesma direção. O mesmo autor descreve o método
\textit{Extreme Programming (XP)} como eficiente, flexível e de baixo risco para equipes
pequenas e médias que convivem com constante mudanças.

Ambos os métodos definem pepeis, práticas e modelos de ciclo de vida para
projetos de desenvolvimentos ágeis e serão utilizados como base para a definição
do método de desenvolvimento deste trabalho.

\subsection{Práticas e Técnicas}
\label{sec:praticas_tecnicas}

São definidas algumas práticas no método \textit{Scrum}~\cite{gutierrez:2009}. A lista a seguir
mostra as que serão utilizadas, bem como a descrição das adaptações para o trabalho:

\begin{itemize}
  \item \textit{\textbf{Sprint}}: ciclos onde são desenvolvidos os itens propostos. Geralmente
    são intervalos de 2 a 4 semanas. Ao final de cada \textit{Sprint} é entregue uma porção 
    executável do \textit{software}. Neste trabalho será utilizado \textit{Sprints} com período
    de 2 semanas;
  \item \textbf{\textit{Product Backlog}}: o \textit{Product Backlog} contém dos os itens a serem
    desenvolvidos no projeto, sendo uma visão macro de tudo a ser feito.
    Com o auxilio da ferramenta Github, os itens dos \textit{backlogs} serão dispostos
    como \textit{issues};
\end{itemize}

O XP traz doze práticas essenciais~\cite{gutierrez:2009}. A lista a seguir
mostra as que serão utilizadas, bem como a descrição das adaptações para o trabalho:

\begin{itemize}
  \item \textbf{Teste}: são divididas em duas partes: teste de aceitação, elaboradas pelo cliente,
    e testes de unidade, elaboradas pelo programador. Será utilizado apenas os testes
    de unidade com a ferramenta RSpec.
  \item \textit{\textbf{Refactoring}}: consiste em simplificar a estrutura, mudar a organização do código,
    sem que altere o comportamento~\cite{beck:2000}. Neste trabalho será utilizado
    conforme as necessidades, sendo levado em consideração a import{\^a}ncia, impactos na
    arquitetura da ferramenta e a prioridade em relação aos outros itens da \textit{Sprint};
  \item \textbf{Integração Contínua}: definido na Seção~\ref{sec:int_con};
  \item \textbf{Programação em Pares}: dois programadores utilizam o mesmo equipamento para
    o desenvolvimento, sendo assim o código está sempre sendo supervisionado por
    outro programador. Técnica utilizada apenas quando necessário, como em uma grande
    alteração da estrutura ou no funcionamento de algum módulo ou classe.
\end{itemize}

\subsection{Controle de Versão}
\label{sec:ctrl_versao}

As principais preocupações da gerência de configuração são~\cite{koskela2003software,}:
\begin{enumerate}
  \item Estabelecer as versões estáveis do sistema ou componente conhecido
    como \textit{baseline};
  \item Possibilitar desfazer modificações no sistema, por conta de erros,
    rejeição do usuário, etc;
  \item Recuperar informações sobre quem e o que foi alterado no sistema e como
    essa alteração está ligada com as necessidades do projeto;
\end{enumerate}

Quanto as ferramentas de suporte a gerência de configuração, existem várias alternativas.
Dentre elas, como mapeado na Seção~\ref{sec:supdev:git}, tem-se o Git que é responsável pelo
controle de versão do sistema. Nela é possível contornar as principais preocupações
descritas acima.

\citeonline{driessen:2010} apresenta um modelo de fluxo de desenvolvimento utilizando a
ferramenta Git. Nele são abordados as estratégias de controle de \textit{branchs} e
gerenciamento de \textit{release}.
