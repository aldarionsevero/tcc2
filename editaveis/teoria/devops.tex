\section{DevOps}
\label{sec:devops}

O termo DevOps tem sido usado com frequência em diversas esferas do
desenvolvimento de software da atualidade, mas por ser um conceito recente
(2008), muita confusão ainda é gerada ao tentar definir e trabalhar com
DevOps~\cite{adambertram:2016}. A palavra DevOps vem de duas palavras em
inglês, \textit{development} e \textit{operations} (desenvolvimento e operações) e de maneira
geral é a cultura, movimento ou conjunto de práticas que incentiva
a comunicação, a colaboração e a integração de desenvolvedores de software
e outros profissionais de TI. Além das práticas também engloba ferramentas
e técnicas que automatizam o processo de entrega de software e as mudanças
de infraestrutura~\cite{loukides2012devops, erich2014mapping}.

Muitas vezes o termo é confundido com uma nova responsabilidade, ou cargo
dentro de uma empresa que desenvolve software, e por mais que seja possível
ter profissionais que tenham proficiência nas ferramentas relacionadas a
DevOps, o ideal, como dito anteriormente, é ter uma melhor comunicação,
colaboração e integração entre os times já existentes. As ferramentas
relacionadas à DevOps facilitam esses aspectos, mas o diferencial é a
mudança no processo de desenvolvimento para absorver essas melhorias
~\cite{adambertram:2016}.

%TODO: parece está faltando mais alguma coisa de DevOps

\section{Metodologia Ágil e DevOps}
\label{sec:agile-devops}

A metodologia ágil surgiu como uma resposta às maneiras tradicionais de desenvolvimento
de software considerando uma nova abordagem com relação as práticas, organização,
documentação e foco no desenvolvimento~\cite{agilemetorg:2016}.

Os métodos ágeis são formas de sustentação da filosofia Ágil proposta no manifesto
Ágil~\cite{fowler:2001}. As duas mais populares são o \textit{Scrum} e o \textit{Extreme Programming}. Nelas,
são definidas práticas que eram comumente utilizadas em outros contextos,
mas foram reunidas e adaptadas para se adequarem à metodologia ágil~\cite{shore:2007}.

Diferentemente do método em cascata, por exemplo, o \textit{Scrum} implementa uma
abordagem iterativa e incremental, podendo assim desenvolver incrementos de
maior valor para o cliente mais cedo, e assim tendo \textit{feedback} para correções
mais frequentes.

Com a popularização da metodologia ágil, que tem, dentre outros
objetivos, o de entregar com maior frequência, e melhorar a comunicação entre os
times, é simples fazer a relação de DevOps com esse tipo de desenvolvimento.
DevOps nada mais é do que a implementação de conceitos e mudanças organizacionais
e culturais provenientes do pensamento Ágil~\cite{scott2014}.

DevOps tenta alcançar entregas mais frequentes ao preparar um ambiente que facilite,
automatize e integre vários dos processos que antes seriam manuais, e mais
suscetíveis à falhas e atrasos, o que não é possível sem uma equipe integrada
nesse ambiente. Dessa forma, o conceito de entrega contínua e de integração
contínua estão fortemente relacionados à DevOps~\cite{adambertram:2016}.

\subsection{Integração Contínua}
\label{sec:int_con}

Integração contínua é a prática de integrar diversas partes de um software
desenvolvido em diversas frentes, de maneira periódica, ou a cada mudança.
Foi adotado como parte do método \textit{Extreme Programming} (XP) que sugere integrar
partes do software mais de uma vez por dia \cite{fowler2006continuous}.

\subsection{Entrega Contínua}
\label{sec:ent_con}

A entrega contínua é uma prática adotada pelos métodos ágeis que tem o objetivo
de preparar um \textit{software} para que ele seja passível de ser posto em produção a
qualquer momento~\cite{olausson:2016}. 

A prática de entrega contínua  é frequentemente confundida com a integração
contínua. Existe uma relação de dependência em que a integração contínua
construi uma estrutura que possa sustentar a entrega contínua de um
\textit{software}.~\citeonline{olausson:2016} resumem os dois em:

\begin{itemize}
  \item \textbf{Integração Contínua}: é voltada para estabelecer uma rápida
    validação da fase de desenvolvimento, ou seja, é criada para o
    desenvolvedor;
  \item \textbf{Entrega Contínua}: é voltada para estabelecer uma cultura onde
    pode-se oferecer um recurso ou funcionalidade para o cliente a qualquer
    momento, ou seja, entregar parte do sistema ao cliente.
\end{itemize}

