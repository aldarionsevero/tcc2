\section{Extração de Configuração}

No processo de automação de uma infraestrutura tem-se a extração das
informações referentes ao sistema. Algumas ferramentas realizam esse
processo como o Ohai~\cite{ohaidoc:2016} e Facter~\cite{facterdoc:2016},
direcionados para o Chef e Puppet, respectivamente. Essas informações
são utilizadas para monitoramento do ambiente controlados pelas ferramentas
de automação.

O método de extração consiste na utilização de outras ferramentas de sistema,
ou seja, a saída da execução de uma ferramenta contém as informações necessárias
a serem extraídas, podendo estar filtradas ou não~\cite{ohaidoc:2016}. O Ohai,
por exemplo, utiliza \textit{plugins} que definem o método de coleta das informações,
a ferramenta de sistema a ser utilizada e o método de filtragem.

Como será visto no Capítulo~\ref{chap:lev_es}, este é o primeiro passo da proposta.
As informações extraídas referem-se às configurações do ambiente. O passo
seguinte é a criação dos \textit{scripts}. Nas pesquisas realizadas foiidentificada
apenas uma ferramenta, nomeada Blueprint, realiza estes passos.

\subsection{Blueprint}

O Blueprint é uma ferramenta de gerência de configuração que realiza a
engenharia reversa do sistema para extrair, em conjunto a outras ferramentas,
as informações de pacotes, serviços e fontes de instalação de aplicativos
~\cite{blueprint:2016}. O Blueprint foi descontinuado em 2014.

A aplicação é utilizada para criar \textit{scripts} que realizam
a instalação dos pacotes e configurações dos serviços que foram extraídos. Há
também a opção de realizar a conversão desses \textit{scripts} para receitas Chef e módulos
Puppet\footnote{Os módulos Puppet funcionam como receitas Chef, mas são específicos para a ferramenta Puppet.}.
Por esse motivo, o Blueprint é considerado como uma ferramenta concorrente ao Cupper,
pois a sua funcionalidade é semelhante.

A Tabela~\ref{tab:cupper-blueprint} mostra uma comparação entre o Cupper e Blueprint.

\begin{table}[]
  \centering
  \caption{Comparativo entre as ferramentas Cupper e Blueprint}
  \label{tab:cupper-blueprint}
  \begin{tabular}{l|l|l}
    \hline
                                                                       & Cupper                                                                                                   & Blueprint                                                                                     \\ \hline
  \begin{tabular}[c]{@{}l@{}}Informações\\ Extraídas\end{tabular}    & \begin{tabular}[c]{@{}l@{}}Pacotes, serviços,\\ configurações, rede,\\ plataforma, hardware\end{tabular} & \begin{tabular}[c]{@{}l@{}}Pacotes, serviços,\\ configurações\end{tabular}                    \\ \hline
  Saídas                                                             & \textit{Cookbook Chef}                                                                                   & \textit{\begin{tabular}[c]{@{}l@{}}Script Shell, Cookbook\\ Chef, Module Puppet\end{tabular}} \\ \hline
  \begin{tabular}[c]{@{}l@{}}Linguagem de\\ Programação\end{tabular} & \textit{Ruby}                                                                                            & \textit{Python}                                                                               \\ \hline
  \begin{tabular}[c]{@{}l@{}}Gerenciador de\\ Pacotes\end{tabular}   & \begin{tabular}[c]{@{}l@{}}APT/dpkg, Pacman,\\ RubyGem e PIP\end{tabular}                                & \begin{tabular}[c]{@{}l@{}}APT, Yum, RubyGems, easy\_install,\\ PIP, PECL e NPM.\end{tabular} \\ \hline
  \end{tabular}
\end{table}
