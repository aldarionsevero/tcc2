
\subsection{\textit{Definitions}}
\label{sec:lev-rec-def}

As \textit{definitions} são um novo tipo de recurso disponível apartir da versão
12.5 do Chef e é recomentado utilizar o \textit{Custom Resource}(Seção \ref{sec:lev-rec-cust})
no lugar de \textit{definitions}~\cite{chefdoc:2016}.

Os \textit{definitions} são comportamento que podem ser reutilizados por outros \textit{recipes}.
São utilizado como recursos padrões de uma receita. No Código \ref{code:definition}
é definido o recurso \textit{host\_porter} com os parametros \textit{port} (valor padrão 4000)
e \textit{hostname} (valor padrão \textit{nil}) que pode ser utilizado em outro
\textit{recipe} com a simples chamada \textit{host\_porter}.

\begin{minipage}{.90\textwidth}
  \lstset{style=shell}
  \lstinputlisting[language=Bash, label=code:definition, caption=Exemplo de \textit{definition}. Adiciona um recurso \textit{host\_porter}.]{editaveis/code/definition_example.rb}
\end{minipage}

Esse atributo não será incluso no escopo de implementação. As \textit{definitions} são
construídas de acordo com as necessidades de cada organização, não sendo
possível prevê-las ou fazer uma \textit{definition} genérica.

