
\subsection{\textit{Templates}}

Os \textit{templates} são arquivos no formato Embedded Ruby (ERB) e permitem gerar
dinamicamente textos de arquivos estáticos. São muito utilizados em arquivos
de configuração que devem ser modificados de acordo com o ambiente. Os
\textit{templates} são inseridos em \textit{COOKBOOK\_NAME/templates} e deve-se declarar o recurso
\textit{template} nos arquivos de \textit{recipes}.

Esse atributo será incluso no escopo de implementação. A leitura de um arquivo de
configuração ou de implantação de uma aplicação será replicado para a
pasta \textit{cookbook/COOKBOOK\_NAME/templates/} de mesmo nome. Os \textit{templates} serão
gerados apartir da coleta de arquivos considerados din{\^a}mico, ou seja,
o conteúdo contem variáveis que estão relacionado ao atributo do ambiente,
por exemplo o IP do ambiente.
