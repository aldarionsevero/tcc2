
\subsection{\textit{Resources e Providers}}
\label{sec:cbresource}

Um \textit{resource} é a definição do passo que deve ser seguido durante o processo
de configuração. Cada \textit{resource} diz ao \textit{chef-client} qual a tarefa a ser executada
como instalar um pacote, criar um arquivo, reiniciar um serviço, etc.
Enquanto o \textit{resource} diz o que deve ser feito, o \textit{provider} diz como deve ser
feito. Por exemplo, o \textit{resource} define \lq\lq instale o pacote A\rq\rq e o \textit{provider} decide
se deve utilizar pacotes deb ou rpm.

Esse atributo será incluso no escopo de implementação. Os \textit{resources} padrão
providos pelo Chef serão utilizados para definir os passos nos arquivos de
\textit{recipes}. Os principais são:

\begin{itemize}
  \item \textit{\textbf{cookbook\_file}}: transfere o arquivo definido em \textit{COOKBOOK\_NAME/files/} para
    uma localização definida (já descrito na Seção \ref{sec:cbfiles});
  \item \textit{\textbf{link}}: usado para criar \textit{links} simbólicos ou real;
  \item \textit{\textbf{package}}: usado para gerenciar pacotes sem a necessidade de especificar a plataforma,
    ou seja, o \textit{provider} irá determinar qual gerenciador de pacote a plataforma utiliza;
  \item \textit{\textbf{service}}: usado para gerenciar serviços da plataforma;
  \item \textit{\textbf{user}}: usado para criar, atualizar, remover e bloquear/desbloquear um usuário do ambiente.
  \item \textit{\textbf{group}}: usado para criar, atualizar, remover um groupo do ambiente.
\end{itemize}
