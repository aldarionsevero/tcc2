
\subsection{\textit{Custom Resource}}
\label{sec:lev-rec-cust}

Adicionado recentemente ao Chef, o \textit{custom resource} é uma forma de criar
novos recursos para os \textit{recipes}. É semelhante as \textit{libraries} e as \textit{definitions},
entretanto é direcionado especificamente para a criação de novos recursos~\cite{chefdoc:2016}.
É uma forma simples de estender o Chef e é implementado dentro de um
\textit{cookbook}.

No Código \ref{code:custom} tem-se a definição de um recurso criado em \textit{cookbooks/app/resources}
com o nome \textit{httpd} e uma propriedade \textit{homepage} com um valor padrão vazio.
Esse recurso é distribuido por todo o \textit{cookbooks} como uma simples chamada
de um recurso Chef como demonstra o Código \ref{code:custom-user}.

\noindent\begin{minipage}{.45\textwidth}
  \lstset{style=shell}
  \lstinputlisting[language=Bash, label=code:custom, caption=Exemplo de declaração de um \textit{custom resource}.]{editaveis/code/custom_example.rb}
\end{minipage}\hfill
\begin{minipage}{.45\textwidth}
  \lstset{style=shell}
  \lstinputlisting[language=Bash, label=code:custom-user, caption=Exemplo de utilização de um \textit{custom resource}.]{editaveis/code/custom_user.rb}
\end{minipage}

Esse atributo não será incluido no escopo, pois a definição dos \textit{custom resource} dependem
da necessidade de cada organização não sendo possível prevê-las.
