\section{Funcionalidade Gerar Diretório de Projeto}
\label{sec:proj}

Essa funcionalidade cria a estrutura de diretório para que o Cupper gere as 
receitas Chef e leia o seu arquivo de configuração, o Cupperfile.

Como exibido no Código~\ref{code:thorsaida2}, o básico dessa funcionalidade é, bem
simples. Um diretório com o nome escolhido vai ser criado, dentro dele um diretório
para as cookbooks, um arquivo oculto para listar arquivos sensíveis, e um 
Cupperfile para as definições que o usuário define para a extração.

\subsection{Definição de Ambiente}
Existem algumas classes importantes no código do Cupper que são relacionadas a
definição do ambiente de execução. Umas delas é a classe do \textit{Cupperfile} e outra
é o \textit{Environment}.

A classe de \textit{Environment} é responsável por checar se o ambiente é um
ambiente válido com a presença de um Cupperfile, definir o diretório do ambiente
(\textit{root\underline{ }path}) e instanciar um \textit{Cupperfile}. A classe \textit{Cupperfile} carrega as configurações definidas no arquivo Cupperfile.

\subsection{Problemas encontrados}
Devido a priorização realizada, e com o andamento do projeto, o carregamento de
configurações adicionais vindas do Cupperfile tinha sido tirado do \textit{backlog} do
trabalho.

Essa decisão seguiu por algum tempo até que algumas falhas fizeram o Cupperfile
ser indispensável para deixar o Cupper funcionando para mais casos.

\subsubsection{Outras Abordagens de Leitura do Cupperfile}
Para a leitura e carregamento das configurações definidas no Cupperfile diversas
abordagens foram consideradas e testadas. A primeira abordagem foi espelhar a
leitura do Cupperfile ao que o Vagrant faz com o Vagrantfile. O Vagrant é outra
ferramenta dentro do contexto de DevOps que inspirou algumas das decisões arquiteturais
do Cupper. Assim como o Vagrant, essa abordagem iria carregar as configurações
do Cupperfile que seria escrito em Ruby. O código do Cupperfile seria carregado
utilizando o \textit{Kernel.load}, que carrega e executa códigos externos.
Essa abordagem foi abandonada pois previa diversas instancias do \textit{loader}
e de objetos que possuiriam informações de vários ambientes (como é necessário
no Vagrant), e isso se mostrou desnecessário e muito complexo para o Cupper.

A segunda abordagem para o carregamento das configurações do Cupperfile foi a de
utilizar uma Gem externa (\textit{onfiguration}) que define um padrão de escrita de 
arquivos de configuração (semelhante a um JSON), e permite o carregamento 
simples deles. Essa também foi abandonada por adicionar mais dependências para o Cupper.

A terceira e quarta abordagens voltaram a tentar utilizar Ruby puro para resolver
esse processo de carregamento de arquivos de configuração, dessa vez tentando 
fazer um módulo mais simples que prevesse somente um ambiente em que o Cupper
iria operar. A quarta abordagem consistiu de somente tornar isso global e
pertencente do módulo \textit{Cupper::Config} acessível em todo o código.

\textit{Links} úteis para carregamento de arquivos de configuração de todas essas
abordagens podem ser encontrados no Apêndice na Seção~\ref{apc:conf-file}.

\subsection{Mudanças e Justificativas}
O escopo tinha sido alterado, e o carregamento de configurações do Cupperfile
tinha sido removido do backlog. Mas esse carregamento se mostrou necessário
e acabou voltando para o desenvolvimento.

Algumas dessas definições do Cupperfile foram reativas, e implementadas de acordo
com o surgimento da necessidade de implementar. Outras tinham sido previstas, mas
não prioritárias, como a de \textit{allow downgrade}.

O que fez essa abordagem voltar a ser indispensável foi o fato de que o Cupper
não pode gerar uma receita que quebra o acesso do usuário a um sistema, ou pior,
que inutilize o sistema.
