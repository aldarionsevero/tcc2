\section{Funcionalidade de Extrair Arquivos de Configuração}
\label{sec:files}

\subsection{Atributos}

Para a funcionalidade mínima desse recurso, é necessário os seguintes
atributos:

\begin{itemize}
  \item \textit{source}: fonte com o conteúdo do arquivo;
  \item \textit{owner}: dono do arquivo, referente ao usuário no ambiente;
  \item \textit{group}: define o grupo ao qual o arquivo pertence;
  \item \textit{mode}: permissões do arquivo;
\end{itemize}

\subsection{\textit{Plugin} de Extração}

A ferramenta Ohai não tem nenhum \textit{plugin} nativo para a extração de arquivos, portanto
foi criado um \textit{plugin} específico de extração de arquivos de configuração. Os arquivos
são aqueles alocados no diretório \texttt{/etc/}. Por padrão das distribuições Linux,
o \texttt{/etc/} armazena os arquivos de configuração específicos do sistema.
Com isso, a extração desses arquivos conterão as configurações de cada pacote e do
sistema. Outras configurações podem estar armazenadas em outros diretórios,
mas o Cupper ainda não provê funcionalidade para recapeara-os. % TODO: Adicionar isso aos trabalhos futuros

Nem todos os arquivos disponíveis no \texttt{/etc/} estão relacionados a um pacote.
Arquivos postos nesse diretório de forma manual, ou seja, sem a utilização de pacotes,
podem estar presentes. Para evitar a coleta desnecessárias desses tipos de arquivos,
o Cupper não extraí arquivos que não estejam relacionados a um pacote.

Nas plataformas Debian \textit{based}, o \textit{plugin} utiliza a ferramenta \texttt{dpkg}
para identificar os arquivos que estão relacionados com um pacote. Nas plataformas
Archlinux \textit{based}, o \textit{plugin} utiliza a ferramenta \texttt{pacman} com o mesmo propósito.
O que difere a extração de uma plataforma para a outra é a interpretação da saída
de cada ferramenta.

As informações de \textit{owner},\textit{group} e \textit{mode} são extraídas com a ferramenta \texttt{ls},
o conteúdo com a ferramenta \texttt{cat} e o tipo com a ferramenta \texttt{file}
que estão presentes em todas as distribuições Linux.
O Código~\ref{code:files_json} mostra um exemplo do JSON gerado pelo \textit{plugin}.

\noindent\begin{minipage}{\textwidth}
  \lstset{style=shell}
  \lstinputlisting[
    label=code:files_json,
    caption="Saída JSON do \textit{plugin} Ohai \textit{files}"]{editaveis/code/json_files.json}
\end{minipage}\hfill

Os arquivos considerados sensíveis, ou seja, com conteúdo sigiloso, como
senhas, chaves ou certificados, podem ser especificados no arquivo de
configuração \texttt{.sensibles} no diretório raiz do projeto Cupper. Com isso,
esses arquivos não serão extraídos, sendo decisão do usuário a melhor forma
de recuperar esses arquivos.

\subsection{Bloco de Receita Gerado}

É utilizado o recurso Chef \textit{cookbook\_file} para cada arquivo extraído.
Os atributos usados na receita gerada são \textit{source}, \textit{owner},
\textit{group} e \textit{mode} como mostra o Código~\ref{code:files_recipe}.

\noindent\begin{minipage}{\textwidth}
  \lstset{style=shell}
  \lstinputlisting[
    label=code:files_recipe,
    caption="Exemplo de receita gerada pela extração de arquivos de configuração"]{editaveis/code/files_recipe.rb}
\end{minipage}\hfill

O conteúdo do arquivo coletado é salvo no diretório \texttt{files} do projeto Cupper
com o caminho descrito no atributo \textit{source}.

\subsection{Problemas Encontrados}

A primeira abordagem da extração dos arquivos de configuração foi a coleta de todos
os arquivos do diretório \texttt{etc}. Era evidente a necessidade de filtragem dos
arquivos, então definimos alguns critérios para eliminar aqueles arquivos que não são
necessários ou deixá-lo como opcional.

Como dito antes, os arquivos sensíveis são aqueles que contem
informações sigilosas ou que pode causar perda de alguma funcionalidade importante.
Durante os testes, o arquivo \textit{sudoers} continha falhas de sintaxe o que causou
quebra completa no sistema. Sendo que este e outros arquivos fazerem parte das
configurações do sistema, decidimos manter a extração deles, mas deixando uma opção
para o usuário em copiá-los ou não para a receita. Esta opção foi adicionada como
atributo no Cupperfile (Código~\ref{code:sensiblefiles})

\noindent\begin{minipage}{\textwidth}
  \lstset{style=shell}
  \lstinputlisting[
    language=Ruby,
    label=code:sensiblefiles,
    caption="Definição do atributo de \textit{sensible files}"]{editaveis/code/sensiblefiles.rb}
\end{minipage}\hfill

Arquivos que não são vinculados a um pacote são descartados. Essa abordagem força
a padronização de ambientes que seguem a instalação e configuração por pacote.

