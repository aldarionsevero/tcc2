\section{Funcionalidade de Extrair Usuários}
\label{sec:users}

\subsection{Atributos}

Para a ter a funcionalidade mínima esperada desse recurso, definimos que iríamos precisar
extrair os seguintes atributos:


\begin{itemize}
    {\itshape\item name}: Identifica nome do usuário;
    {\itshape\item uid}: Identificador do usuário;
    {\itshape\item home}: Identificar diretório \textit{home} desse usuário;
    {\itshape\item shell}: Identificar qual \textit{shell} o usuário usa ;
    {\itshape\item manage\_home}: Identificar se a \textit{home} deve ser criada;
\end{itemize}

\subsection{\textit{Plugin} de Extração}
Não foi necessário desenvolver um \textit{plugin} para extrair usuários pois o Ohai
já possui um \textit{plugin} nativo para isso. Ele provê atributos referentes
a informações de usuários, grupos, permissões entre outras.

Esse \textit{plugin} utiliza o módulo Ruby \textit{etc} que provê acesso a informações
do sistema operacional, diferentemente de alguns outros \textit{plugins} do Ohai
que executam comandos \textit{shell} diretamente na máquina. A saída é interpretada
pelo \textit{plugin} e transformada em um JSON construído com a estrutura
\texttt{``etc''=> \{ ``passwd''=> \{ ``root''=> \{ ``dir''=> ``/root'', ``gid''=> 0, ``uid''=> 0, ``shell''=> ``/bin/bash'', ``gecos''=> ``root'' \}} 
como mostrada no Código~\ref{code:json_user}

\noindent\begin{minipage}{\textwidth}
  \lstset{style=shell}
  \lstinputlisting[frame=single,
    label=code:json_user,
    caption="Saída JSON do \textit{plugin} Ohai \textit{passwd} para usuários"]{editaveis/code/json_user.json}
\end{minipage}\hfill

\subsection{Bloco de Receita Gerado}

O JSON do Código~\ref{code:json_user} é usado para construir a receita responsável pela
criação dos usuários. O recurso Chef \textit{user} é usado para essa criação com
os atributos \textit{name}, \textit{uid}, \textit{home}, \textit{shell} e 
\textit{manage\_home}, como pode ser observado no Código~\ref{code:user_recipe}.

\noindent\begin{minipage}{\textwidth}
  \lstset{style=shell}
  \lstinputlisting[
    language=Ruby,
    frame=single,
    label=code:user_recipe,
    caption="Exemplo de receita gerada pela extração de usuários."]{editaveis/code/user_recipe.rb}
\end{minipage}\hfill

\subsection{Problemas Encontrados}
A receita não conseguia ser executada em um novo ambiente, ou quando conseguia
ser executada, perdia as referências entre os grupos e os usuários. Isso fazia
com que alguns usuários perdessem privilégios de super usuário, ou que não
conseguissem executar nenhum comando.

Para solucionar esse problema uma funcionalidade de extração de grupos, não prevista
no escopo inicial, teve que ser implementada. Sua descrição está na Seção~\ref{sec:groups}.
