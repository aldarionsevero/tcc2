\chapter{Resultados}
\label{chap:result}

Neste capítulo serão abordados os resultados dos testes realizados
para validar a capacidade do Cupper de reproduzir um ambiente.

Os testes foram realizados em dois tipos de ambientes ao longo do
desenvolvimento do projeto. Os ambientes controlados são máquinas
virtuais locais. Foi utilizado a ferramenta Vagrant e VirtualBox
como virtualizadores e as distros disponíveis no repositório Hashicorp
\footnote{Site oficial que contem todos os tipos de distros em formato \textit{box} para o Vagrant. Disponível em https://atlas.hashicorp.com/boxes/search.}.

Os ambientes reais são máquinas virtuais disponíveis na infraestrutura
do LAPPIS (Laboratório Avançado de Produção Pesquisa e Inovação em Software).
Tais máquinas contem aplicações, serviços e configurações em modo produção.

As próximas sessões irão mostrar o resultado de cada ambiente testado,
a quantidade de pacotes, configurações e serviços que foram replicados
e relatos de problemas ou abordagens.

\section{Premissas e Pré-requisitos}

Os testes podem ser divididos em dois momentos: execução do Cupper no
ambiente alvo e execução da receita gerada pelo Cupper. Portanto,
para os testes, e utilização correta do Cupper, algumas premissas
e pré-requisitos devem ser alcançados.

No primeiro momento, para a execução do Cupper, o ambiente alvo deve
ser uma distribuição Debian \textit{based}, deve conter Ruby na versão
2.1.0 ou superior, deve ter as dependências para instalação
de Ruby \textit{Gems} e o usuário deve ter permissão de execução do Cupper, assim como
permissão de criar pastas e arquivos.

No segundo momento, para execução da receita gerada pelo Cupper, o novo
ambiente deve ser uma distribuição Debian \textit{based} na mesma versão do ambiente
alvo, deve ter acesso aos mesmos repositórios de pacotes que
o ambiente alvo, deve ter acesso a Internet, deve estar configurado com o Chef Client e
o usuário deve ter permissão de instalação de pacotes, criação e alteração de arquivos no
diretório \texttt{/etc}.

Utilizou-se a ferramenta Chake para os testes no segundo momento. O Chake
permite a execução do Chef em um ambiente remoto sem a necessidade de um
Chef Server. O \textit{cookbook} gerado pelo Cupper pode ser utilizado em qualquer
ambiente que utilize o Chef, seja Chef Solo, Chef Zero ou Chef Server.

\section{Ambientes Controlados}
\subsection{Serviço Cups}
O ambiente controlado que foi montado para testes iniciais do Cupper foi planejado
para ser simples e conter um serviço com poucas dependências para validar um caso
sem complicações. Diferentemente dos testes tratados a seguir nas próximas Seções,
esse ambiente não se trata de um ambiente real de uso, e foi criado somente para
esse fim.

A máquina virtual utilizada para esse teste foi montada com a distro Debian 8.
O CUPS, que é um serviço de gerenciamento de impressões criado pela Apple foi
instalado nesse ambiente base, juntamente com suas dependências e nginx. A instalação
do CUPS já levanta o seu serviço web que fica ativo no host local na porta 631.
Além de instalar o CUPS, um arquivo de configuração para o CUPS foi criado
para verificar se essa configuração seria passada para o novo ambiente. O
conteúdo desse arquivo pode ser visto no Código~\ref{code:cupsconfig}. Outro 
arquivo de configuração para o Nginx foi definido para efetuar o 
\textit{proxypass} da porta 631 para a porta 80 (Código~\ref{code:nginxconfig}), fazendo assim pelo menos duas
aplicações se relacionarem.

O novo ambiente foi montado com a mesma imagem usada ná máquina virtual já citada.
Os pacotes já instalados e serviços ativos, antes de executar a receita, foram somente 
os que já vem por padrão na distribuição.

O Cupper foi executado com as opções padrão do Cupperfile, e a Tabela~\ref{tab:result_cups}
mostra os resultados alcançados após gerar a receita no ambiente base e executá-la
no ambiente novo.


\begin{table}[H]
  \centering
  \caption{Resultados da máquina virtual com o serviço Moodle}
  \label{tab:result_cups}
  \begin{tabular}{c|c|c|c}
    \hline
    \rowcolor[HTML]{EFEFEF} 
    {\color[HTML]{000000} \textbf{Camada}} & {\color[HTML]{000000} \textbf{Ambiente}}                                       & {\color[HTML]{000000} \textbf{Ambiente Replicado}}                             & {\color[HTML]{000000} \textbf{Replicado}} \\ \hline
                                           & \begin{tabular}[c]{@{}c@{}}Dependências do CUPS\\ Instaladas\end{tabular}        & \begin{tabular}[c]{@{}c@{}}Dependências do CUPS\\ Instaladas\end{tabular}        & X                                         \\ \cline{2-4} 
                                           & \begin{tabular}[c]{@{}c@{}}Dependências do Nginx\\ Instaladas\end{tabular}                & \begin{tabular}[c]{@{}c@{}}Dependências do Nginx\\ Instaladas\end{tabular}                & X                                         \\ \cline{2-4} 
                                           & \begin{tabular}[c]{@{}c@{}}Pacote CUPS\\ Instalado\end{tabular} & \begin{tabular}[c]{@{}c@{}}Pacote CUPS\\ Instalado\end{tabular} & X                                         \\ \cline{2-4} 
  \multirow{-6}{*}{Aplicação}            & \begin{tabular}[c]{@{}c@{}}Pacote Nginx\\ Instalado\end{tabular}           & \begin{tabular}[c]{@{}c@{}}Pacote Nginx\\ Instalado\end{tabular}       &                                          X \\ \hline
                                           & \begin{tabular}[c]{@{}c@{}}Serviço Apache\\ Executando\end{tabular}            & \begin{tabular}[c]{@{}c@{}}Serviço Apache\\ Executando\end{tabular}            & X                                         \\ \cline{2-4} 
  \multirow{-2}{*}{Serviços}             & \begin{tabular}[c]{@{}c@{}}Serviço CUPS\\ Executando\end{tabular}             & \begin{tabular}[c]{@{}c@{}}Serviço CUPS\\ Executando\end{tabular}             & X                                         \\ \hline
                                           & \begin{tabular}[c]{@{}c@{}}Configurações CUPS\\ Carregadas\end{tabular}            & \begin{tabular}[c]{@{}c@{}}Configurações CUPS\\ Carregadas\end{tabular}            & X                                         \\ \cline{2-4} 
  \multirow{-2}{*}{Configurações}             & \begin{tabular}[c]{@{}c@{}}Configurações Nginx\\ Carregadas\end{tabular}             & \begin{tabular}[c]{@{}c@{}}Configurações Nginx\\ Carregadas\end{tabular}             & X                                         \\ \hline
  \end{tabular}
\end{table}

No ambiente base em que o CUPS e o Nginx foram instalados manualmente, o CUPS
podia ser acessado na porta 631 do \textit{host} local e o Nginx estava fazendo
com que também fosse possível acessar o CUPS na porta 80. Após a execução da
receita gerada, o novo ambiente passou a ter o mesmo comportamento, e dessa
forma o ambiente foi replicado com sucesso.


\section{Ambientes Reais}
\subsection{Serviço \textit{MySQL}}

A máquina que contém o banco MySQL, que é usado por diversas outras máquinas da
infraestrutura do Lappis, foi montada com a distro Debian 8 e está em operação 
desde Janeiro de 2016.

A extração com o Cupper foi completa, com todos os serviços e pacotes extraídos
e replicados no ambiente novo. Todavia, é importante frisar que o conteúdo dos bancos
da máquina não é replicado, sendo de responsabilidade do usuário fazer o \textit{backup}
dos bancos para passar para o ambiente novo.

O novo ambiente foi montado com a mesma imagem da máquina MySQL em produção.
Dessa forma, a distribuição, versão de Kernel e repositórios de pacotes são as
mesmas.

O Cupper foi executado com a opção de versionamento de pacotes desativada, ou seja,
os pacotes instalados no novo ambiente seriam as mais recentes disponíveis. 
O motivo para a abordagem é a indisponibilidade dos pacotes nas versões 
instaladas no ambiente MySQL em produção. Os resultados da extração e replicação
podem ser vistos na Tabela~\ref{tab:result_mysql}.


\begin{table}[H]
  \centering
  \caption{Resultados da máquina virtual com o serviço Moodle}
  \label{tab:result_mysql}
  \begin{tabular}{c|c|c|c}
    \hline
    \rowcolor[HTML]{EFEFEF} 
    {\color[HTML]{000000} \textbf{Camada}} & {\color[HTML]{000000} \textbf{Ambiente}}                                       & {\color[HTML]{000000} \textbf{Ambiente Replicado}}                             & {\color[HTML]{000000} \textbf{Replicado}} \\ \hline
                                           & \begin{tabular}[c]{@{}c@{}}Dependências do MySQL\\ Instaladas\end{tabular} & \begin{tabular}[c]{@{}c@{}}Dependências do MySQL\\ Instaladas\end{tabular} & X                                         \\ \cline{2-4} 
                                           & \begin{tabular}[c]{@{}c@{}}Pacote mysql-client\\ Instalado\end{tabular}        & \begin{tabular}[c]{@{}c@{}}Pacote mysql-client\\ Instalado\end{tabular}        & X                                         \\ \cline{2-4} 
  \multirow{-6}{*}{Aplicação}            & \begin{tabular}[c]{@{}c@{}}Pacote mysql-server\\ Instalado\end{tabular}           & \begin{tabular}[c]{@{}c@{}}Pacote mysql-server\\ Instalado\end{tabular}       & X                                          \\ \hline
  \multirow{-2}{*}{Serviços}             & \begin{tabular}[c]{@{}c@{}}Serviço Mysql\\ Executando\end{tabular}             & \begin{tabular}[c]{@{}c@{}}Serviço Mysql\\ Executando\end{tabular}             & X                                         \\ \hline
  \end{tabular}
\end{table}


\subsection{Serviço Moodle}

A máquina que contem o Moodle foi montada com a distro Ubuntu 14.04 e
está em operação desde Setembro de 2016. O principal serviço, o Moodle,
é usado nas disciplinas de Computação Básica.
A extração com o Cupper foi parcialmente completa, sendo a receita de \textit{links}
gerada continha erro de sintaxe. Ao replicar em um novo ambiente, essa receita
foi descartada.

O novo ambiente foi montado com a mesma imagem que foi utilizada na máquina
Moodle em produção. Com isso, a distribuição, versão de Kernel e repositórios
de pacotes são as mesmas do ambiente de produção. Esse novo ambiente, não
continha nenhum serviço ou pacotes instalados que não fossem os padrões
disponível na distribuição.

A Tabela~\ref{tab:result_moodle} mostra os principais serviços e pacotes instalados
na máquina em produção e os resultados dos mesmos no ambiente replicado.

\begin{table}[H]
  \centering
  \caption{Resultados da máquina virtual com o serviço Moodle}
  \label{tab:result_moodle}
  \begin{tabular}{c|c|c|c}
    \hline
    \rowcolor[HTML]{EFEFEF} 
    {\color[HTML]{000000} \textbf{Camada}} & {\color[HTML]{000000} \textbf{Ambiente}}                                       & {\color[HTML]{000000} \textbf{Ambiente Replicado}}                             & {\color[HTML]{000000} \textbf{Replicado}} \\ \hline
                                           & \begin{tabular}[c]{@{}c@{}}Pacote apache2\\ Instalado\end{tabular}             & \begin{tabular}[c]{@{}c@{}}Pacote apache2\\ Instalado\end{tabular}             & X                                         \\ \cline{2-4} 
                                           & \begin{tabular}[c]{@{}c@{}}Pacote mysql-client\\ Instalado\end{tabular}        & \begin{tabular}[c]{@{}c@{}}Pacote mysql-client\\ Instalado\end{tabular}        & X                                         \\ \cline{2-4} 
                                           & \begin{tabular}[c]{@{}c@{}}Pacote mysql-server\\ Instalado\end{tabular}        & \begin{tabular}[c]{@{}c@{}}Pacote mysql-server\\ Instalado\end{tabular}        & X                                         \\ \cline{2-4} 
                                           & \begin{tabular}[c]{@{}c@{}}Pacote php5\\ Instalado\end{tabular}                & \begin{tabular}[c]{@{}c@{}}Pacote php5\\ Instalado\end{tabular}                & X                                         \\ \cline{2-4} 
                                           & \begin{tabular}[c]{@{}c@{}}Pacote libapache2-mod-php5\\ Instalado\end{tabular} & \begin{tabular}[c]{@{}c@{}}Pacote libapache2-mod-php5\\ Instalado\end{tabular} & X                                         \\ \cline{2-4} 
  \multirow{-6}{*}{Aplicação}            & \begin{tabular}[c]{@{}c@{}}Aplicação Moodle\\ Instalada\end{tabular}           & \begin{tabular}[c]{@{}c@{}}Aplicação Moodle\\ Não Instalada\end{tabular}       &                                           \\ \hline
                                           & \begin{tabular}[c]{@{}c@{}}Serviço Apache\\ Executando\end{tabular}            & \begin{tabular}[c]{@{}c@{}}Serviço Apache\\ Executando\end{tabular}            & X                                         \\ \cline{2-4} 
  \multirow{-2}{*}{Serviços}             & \begin{tabular}[c]{@{}c@{}}Serviço Mysql\\ Executando\end{tabular}             & \begin{tabular}[c]{@{}c@{}}Serviço Mysql\\ Executando\end{tabular}             & X                                         \\ \hline
  \end{tabular}
\end{table}

O Cupper foi executado com a opção de versionamento de pacotes desativada,
ou seja, os pacotes que iriam ser instalados no novo ambiente seriam os
mais recentes disponíveis. O motivo para a abordagem é a indisponibilidade
dos pacotes nas versões instaladas no ambiente Moodle em produção.

A aplicação Moodle não foi replicada no novo ambiente, isso ocorreu porque,
no ambiente de produção, a instalação foi manual, ou seja, foi feito o download dos
arquivos fontes da aplicação e as configurações manualmente inseridas comando por
comando no ambiente.

É importante ressaltar que, os dados presentes no banco da aplicação não são
extraídos, sendo isso uma responsabilidade do usuário em realizar o \textit{backup} dos
dados da forma que considerar mais apropriada.

\subsection{Serviço Portal}

A máquina que contem o Portal FGA foi montada com a distro Debian 8 Jessie
e está em operação desde Novembro de 2016. O principal serviço, o site Portal FGA,
é provido pela plataforma Noosfero. A extração com o Cupper foi completa,
sem erros de sintaxe.

O novo ambiente foi montado com a mesma imagem que foi utilizada na máquina
Portal em homologação. Com isso, a distribuição, versão de Kernel e repositórios
de pacotes são as mesmas do ambiente de homologação. Esse novo ambiente, não
continha nenhum serviço ou pacotes instalados que não fossem os padrões
disponível na distribuição.

A Tabela~\ref{tab:result_portal} mostra os principais serviços e pacotes instalados
na máquina em homologação do Portal.

\begin{table}[H]
  \centering
  \caption{Resultados da máquina virtual com o serviço Portal FGA}
  \label{tab:result_portal}
  \begin{tabular}{c|c|c|c}
    \hline
    \rowcolor[HTML]{EFEFEF} 
    {\color[HTML]{000000} \textbf{Camada}}               & {\color[HTML]{000000} \textbf{Ambiente}}                                    & {\color[HTML]{000000} \textbf{Ambiente Replicado}}                                    & {\color[HTML]{000000} \textbf{Replicado}} \\ \hline
                                                         & \begin{tabular}[c]{@{}c@{}}Pacote noosfero\\ Instalado\end{tabular}         & \begin{tabular}[c]{@{}c@{}}Pacote noosfero\\ Instalado\end{tabular}                   & X                                         \\ \cline{2-4} 
                                                         & \begin{tabular}[c]{@{}c@{}}Pacote postgresql\\ Instalado\end{tabular}       & \begin{tabular}[c]{@{}c@{}}Pacote postgresql\\ Instalado\end{tabular}                 & X                                         \\ \cline{2-4} 
                                                         & \begin{tabular}[c]{@{}c@{}}Pacote portal-unb-theme\\ Instalado\end{tabular} & \begin{tabular}[c]{@{}c@{}}Pacote portal-unb-theme\\ Instalado\end{tabular}           & X                                         \\ \cline{2-4} 
  \multirow{-4}{*}{Aplicação}                          & \begin{tabular}[c]{@{}c@{}}Pacote nginx\\ Instalado\end{tabular}            & \begin{tabular}[c]{@{}c@{}}Pacote nginx\\ Instalado\end{tabular}                      & X                                         \\ \hline
                                                         & \begin{tabular}[c]{@{}c@{}}Serviço Nginx\\ Executando\end{tabular}          & \begin{tabular}[c]{@{}c@{}}Serviço Nginx\\ Executando\end{tabular}                    & X                                         \\ \cline{2-4} 
  \multirow{-2}{*}{Serviços}                           & \begin{tabular}[c]{@{}c@{}}Serviço Noosfero\\ Não Executado\end{tabular}       & \begin{tabular}[c]{@{}c@{}}Serviço Noosfero\\ Executando\end{tabular}                 &                                          \\ \hline
  \multicolumn{1}{l|}{}                                & \begin{tabular}[c]{@{}c@{}}Configurações Noosfero\\ Carregadas\end{tabular} & \begin{tabular}[c]{@{}c@{}}Configurações Noosfero\\ Carregadas Parcialmente\end{tabular}           & X                                         \\ \cline{2-4} 
  \multicolumn{1}{l|}{\multirow{-2}{*}{Configurações}} & \begin{tabular}[c]{@{}c@{}}Configurações Nginx\\ Carregadas\end{tabular}    & \begin{tabular}[c]{@{}c@{}}Configurações Nginx\\ Carregadas Parcialmente\end{tabular} & \multicolumn{1}{l}{}                      \\ \hline
  \end{tabular}
\end{table}

O Cupper foi executado com a opção de versionamento de pacotes desativada. O motivo para
a abordagem é a indisponibilidade dos pacotes nas versões instaladas no ambiente
Portal em homologação. Entretanto o problema persistiu durante a configuração
do novo ambiente, por conta de alguns pacotes não seguirem o padrão de versionamento.
Em geral, os pacotes são instalados especificando a versão: pacote A, versão 1, mas os
pacotes instalados seguia outro padrão: pacote A1, versão 1. Neste caso,
para fazer o \textit{upgrade} do pacote A, é necessário instalar outro pacote: pacote A2,
versão 2. Sendo assim, a receita tentava instalar alguns pacotes em duas versões
diferentes, uma disponível no repositório (atualizada), e outra que não estava mais
no repositório.

Ainda com relação aos pacotes, outro problema foi encontrado. Para instalar pacotes
de um repositório, é necessário uma chave de autenticação para garantir que o repositório
é seguro. O novo ambiente não tem tais chaves, o que causa erro na instalação desses
pacotes.

A ordem dos pacotes também causou erro na execução da receita. Por padrão, a lista de
pacotes extraída está em ordem alfabética, assim colocando o pacote do Noosfero antes
do pacote PostgreSQL. É necessário que o banco de dados esteja em operação para a
configuração do Noosfero.

Tendo em vista esses três problemas, alterações na receita foram necessárias:
remoção dos pacotes sem repositório associado, adição da flag \texttt{allow-unauthenticated}
na instalação dos pacotes não autenticados e modificar a posição do pacote Noosfero
para executar após o pacote PostgreSQL.

A configurações do serviço Nginx foram parcialmente replicadas. Arquivos de configurações
não estavam vinculados ao pacotes. Para esse caso, esses arquivos estavam corretamente
inseridos no diretório \texttt{/etc/nginx/conf.d} destinado a configurações específicas
vinculadas a um escopo. Com isso, é necessário que o Cupper reconheça esses tipos
de arquivos de configuração, não apenas aqueles vinculados aos pacotes.

O serviço do Noosfero na máquina em homologação não estava em execução. Isso poderia
comprometer a configuração do novo ambiente, entretanto não houve problemas, devido
ao erro da aplicação em homologação está referente ao banco de dados. O Cupper não
prevê recuperação do banco de dados, o que não reproduziu o erro no novo ambiente.




