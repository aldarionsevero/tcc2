\section{Contexto}
\label{sec:contexto}

O mercado de desenvolvimento de software, atualmente, pressiona as organizações
a buscarem modelos que ofereçam uma constante entrega de produto com
um intervalo cada vez menor de tempo. Como parte deste cenário, tem-se o
tradicional problema entre a equipe de desenvolvimento e a equipe de operação
~\cite{hummer:2013}.

Enquanto a equipe de desenvolvimento tende a disponibilizar novas alterações,
solicitadas pelo cliente ou providas pela empresa para a validação do cliente o
mais depressa possível, a equipe de operação tende a manter o sistema estável,
o que significa minimizar a implantação de grandes mudanças. A lacuna desse
processo é agravada pelos diferentes objetivos de cada equipe
~\cite{huttermann:2012}.

A adoção de DevOps \textit{(Development and Operating)} vem sendo feita
para amenizar esse problema. DevOps consiste em uma série de práticas,
ferramentas ou mesmo cultura organizacional com o objetivo de diminuir o tempo
de entrega e implantação de um software. Juntamente a esse modelo,
tem-se os conceitos de automação e infraestrutura
como código. Ambos estão fortemente ligados, sendo facilitadores para o
desenvolvedor conhecer as regras de implantação de sua aplicação
e para o operador documentar e configurar um ambiente para um estado específico
definido pelo código de implantação~\cite{hummer:2013}.

Neste contexto, surgiram ferramentas de automação, como Chef~\cite{chef:2016} e
Puppet~\cite{puppet:2016}, que abstraem os passos de execução de configuração
e implantação tranformando-os em códigos simples. A chave dessas propostas está em torno
da capacidade de \textit{convergence} do ambiente, ou seja, a execução de uma
sequência de passos pré-definidos sempre leva a um estado previsível do objeto
~\cite{hummer:2013}. Pode-se citar alguns dos principais benefícios
~\cite{vasiliev:2014}:

\begin{itemize}
  \item \textbf{Escalabilidade}: aumenta a quantidade de serviços em uma infraestrutura
    utilizando \textit{environments} (variáveis do ambientes),\textit{roles}
    (conjunto de configurações que representa o papel do sistema) e \textit{nodes}
    (máquina física, virtual, container, etc);
  \item \textbf{Reuso e Replicação}: possibilita configurar a mesma aplicação em um novo \textit{node}, ou seja,
    conseguir replicar as mesmas configurações pela infraestrutura em pouco tempo;
  \item \textbf{Documentação}: um \textit{script}(ou receita) em Chef contém todas as instruções necessárias
    para configuração do ambiente em formato simples de leitura;
\end{itemize}

Neste cenário, este trabalho visa uma oportunidade de melhoria com relação
a criação dos \textit{scripts} nos padrões da ferramenta de automação Chef, criando uma
ferramenta, denominada Cupper, que realiza a engenharia reversa das configurações
de um ambiente para a criação de \textit{scripts} Chef a fim de facilitar na replicação,
documentação e migração da infraestrutura.

