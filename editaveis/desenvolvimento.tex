\chapter{Desenvolvimento}
\label{chap:dev}

Esse capítulo irá descrever o desenvolvimento e evolução de cada uma das funcionalidades
previstas para o Cupper. Além disso, problemas encontrados durante o processo
de desenvolvimento serão relatados, e mudanças explicadas.

A maior e mais importante funcionalidade é a de gerar a receita Chef propriamente dita.
Dessa forma quebramos essa grande funcionalidade em funcionalidades menores que refletem cada
um dos tipos de extração e reprodução de recursos Chef.

As funcionalidades previstas para o Cupper são as seguintes:

\begin{enumerate}
  \item \textbf{Extract Packages}: Extrair atributos referentes aos pacotes
instalados e gerar receita Chef que instale os mesmos pacotes em um novo ambiente;
  \item \textbf{Extract Links}: Extrair atributos referentes a links simbólicos
de arquivos e gerar receita Chef que reproduza os mesmos links em um ambiente novo;
  \item \textbf{Extract Users}: Extrair atributos referentes a usuários presentes
no ambiente sendo extraído e gerar receita que crie os mesmos usuários em um
ambiente novo.
  \item \textbf{Extract Cookbook Files}: Extrair atributos referentes aos arquivos
de configuração padrão e gerar receita Chef que copie esses mesmos arquivos para o ambiente novo;
  \item \textbf{Extract Groups}: Extrair atributos referentes aos grupos de usuários
e gerar receita Chef que crie e associe os usuários certos a esses grupos em um
ambiente novo;
  \item \textbf{Extract Services}: Extrair atributos referentes aos serviços carregados
e ativos e gerar uma receita Chef que suba os mesmos serviços;
  \item \textbf{Gerar Projeto Cupper}: Gerar a estrutura de diretório padrão 
para o projeto Cupper.
  \item \textbf{Listar Plugins}: Listar os plugins disponíveis do Ohai, tanto 
os padrões, já inclusos na instalação do Ohai, quanto os customizados pelo Cupper;
  \item \textbf{Help}: Imprime no terminal os principais comandos e as opções
válidas de cada uma;
  \item \textbf{Relatório Intermediário}: Gerar relatório intermediário antes
da receita Chef em si, para diagnóstico do sistema sendo extraído e para debug
do Cupper. \textbf{Essa funcionalidade não foi implementada.}

\end{enumerate}

\section{Funcionalidade de Extrair Pacotes}
\label{sec:pacotes}

\subsection{Atributos}
Para ter a funcionalidade mínima esperada desse recurso, definimos que iríamos
precisar extrair os seguintes atributos:

\begin{itemize}
  \item \textit{version}: especifica a versão do pacote;
  \item \textit{action}: define a ação que deve ser tomada
    (instalar, remover, reconfigurar, etc);
\end{itemize}

\subsection{\textit{Plugin} de Extração}

O Ohai provê um \textit{plugin} nativo que coleta todos os pacotes instalados. O suporte
das plataformas são: Debian, Redhat, Fedora e OpenSUSE\@. Os dados coletados pelo
\textit{plugin} de cada pacote são: nome do pacote e versão.

Para a plataforma Debian, o \textit{plugin} utiliza a ferramenta \texttt{dpkg-query} disponível
por padrão nas distro Debian \textit{based}. O \texttt{dpkg-query} retorna uma lista de todos os pacotes
instalados no ambiente que é interpretada pelo \textit{plugin} e transformada em um JSON
construído com a estrutura \texttt{``package\_name'' => \{ ``version'' => ``version\_number'' \}}
como mostrada no Código~\ref{code:json_pkg}.

\noindent\begin{minipage}{\textwidth}
  \lstset{style=shell}
  \lstinputlisting[frame=single,
    label=code:json_pkg,
    caption="Saída JSON do \textit{plugin} Ohai \textit{packages}"]{editaveis/code/json_pkg.json}
\end{minipage}\hfill

Para as plataformas Redhat, Fedora e OpenSUSE, o \textit{plugin} utiliza a ferramenta \texttt{rpm}
disponível por padrão nas distros Redhat \textit{based}. O Cupper ainda não oferece suporte
para distros Redhat \textit{based}, portanto não foram feitos testes para esse tipo de ambiente.

Não há suporte para a plataforma Archlinux, portanto um \textit{plugin} foi desenvolvido
para extrair as mesmas informações. A extração utiliza a ferramenta \texttt{pacman} disponível
nas distros Archlinux. O \texttt{pacman} retorna uma lista de pacotes instalados no ambiente
que é interpretada pelo \textit{plugin} e transformada em um JSON construído com a estrutura
\texttt{"package\_name" => \{ "version" => "version\_number" \}} como mostrada no
Código~\ref{code:json_pkg_pacman}.

\noindent\begin{minipage}{\textwidth}
  \lstset{style=shell}
  \lstinputlisting[frame=single,
    label=code:json_pkg_pacman,
    caption="Saída JSON do \textit{plugin} Ohai \textit(pacman)"]{editaveis/code/json_pkg_pacman.json}
\end{minipage}\hfill

\subsection{Bloco de Receita Gerado}

O JSON gerado pela extração é utilizado para construir a receita responsável pela instalação dos
pacotes. É utilizado o recurso Chef \textit{package} para instalação de cada pacote com os dois atributos
\textit{version} e \textit{action} (Código~\ref{code:pkg_recipe}). \textit{Action} é colocado \textit{\textit{:install}}
como padrão para todos os pacotes.

\noindent\begin{minipage}{\textwidth}
  \lstset{style=shell}
  \lstinputlisting[
    language=Ruby,
    frame=single,
    label=code:pkg_recipe,
    caption="Exemplo de receita gerada pela extração de pacotes."]{editaveis/code/pkg_recipe.rb}
\end{minipage}\hfill

Considera-se que o pacote irá cuidar da suas próprias dependências quanto à instalação.
Sendo assim, os pacotes que são dependências de outros não são inclusos na receita. Exemplo, se o pacote
\textbf{A} contém as dependências \textbf{B} e \textbf{C} elas não serão inclusas na receita, pois o pacote \textbf{A} irá realizar a instalação.

\subsection{Problemas Encontrados}

O método de extração dos pacotes instalados não realiza nenhuma checagem para
saber se o pacote continua disponível no repositório de pacotes remotos ou
se ao menos tem um repositório remoto associado. Isso causou falha na realização
dos testes. Tanto os pacotes instalados manualmente, por meio de download do
\textit{source}, quanto os pacotes que não estão mais disponíveis no repositório remoto ou
mesmo repositórios inseridos manualmente, não eram identificados durante a
configuração do novo ambiente.

Esse comportamento não é esperado, portanto algumas medidas foram tomadas para
que o problema fosse parcialmente corrigido. Para o caso de repositórios que
tenha outro \textit{source} além do padrão Debian, a receita de package gerada adiciona
o atributo \textit{cookbook\_file} contendo o arquivo \texttt{source.list} e outros disponíveis
no diretório \texttt{/etc/apt/source.list.d/} extraído do ambiente. Isso permite
que todos os repositórios não oficiais do ambiente alvo sejam aplicados ao novo ambiente.

Uma possível forma de contornar problema de pacotes sem repositório é
a criação de um atributo para que o usuário defina no Cupperfile se
será extraído pacotes que tem fonte em repositórios fora do \textit{source list} padrão ou
nenhum repositório vinculado.

Outro problema encontrado está relacionado a versão dos pacotes.
A receita especifica para todos os pacotes a versão que foi extraída do ambiente.
Ao realizar a configuração do novo ambiente, é possível que o pacote já esteja instalado,
mas a sua versão esteja à frente da definida na receita. Quando isso ocorre, dois comportamentos
podem ocorrer: não há erro e é feito o \textit{downgrade} do pacote ou ocorre erro no \textit{downgrade} das
dependências do pacote. Isso ocorre porque o gerenciador de pacotes não realiza o \textit{downgrade}
das dependências, e as versões antigas dos pacotes podem exigir as versões antigas das suas
dependências.

Foi criado o Cupperfile, que é um arquivo de configuração de extração que define atributos
que são usados para restringir ou permitir mais atributos ou recursos na receita gerada.
Então para a resolução do problema acima, é colocar um atributo de configuração de extração no
Cupperfile que possibilita o usuário definir se a receita gerada force a instalação de uma
versão antiga. A forma de realizar esse \textit{downgrade} é eliminando o pacote e suas dependências
e instalando a versão mais antiga.

Ainda com relação a versão do pacote, outro caso de erro foi encontrado.
Por conta do sistema estar desatualizado, ou seja, não houve atualização dos pacotes
instalados, a versão instalada pode não estar mais disponível no repositório principal.
Neste caso o gerenciado de pacotes não encontrará a fonte e acusará erro.

Uma solução parcial usada para o problema de versão foi a criação de atributo no
Cupperfile que define se será extraído as informações de versão ou não do pacote.
Com esse atributo definido como falso, a receita gerada não especifica a versão
do pacote, instalando a última versão disponível no repositório.

\section{Funcionalidade de Extrair Arquivos de Configuração}
\label{sec:files}

\subsection{Atributos}

Para a funcionalidade mínima desse recurso, é necessário os seguintes
atributos:

\begin{itemize}
  \item \textit{source}: fonte com o conteúdo do arquivo;
  \item \textit{owner}: dono do arquivo, referente ao usuário no ambiente;
  \item \textit{group}: define o grupo ao qual o arquivo pertence;
  \item \textit{mode}: permissões do arquivo;
\end{itemize}

\subsection{\textit{Plugin} de Extração}

A ferramenta Ohai não tem nenhum plugin nativo para a extração de arquivos, portanto
foi criado um plugin específico de extração de arquivos de configuração. Os arquivos
são aqueles alocados no diretório \texttt{/etc/}. Por padrão das distribuições Linux,
o \texttt{/etc/} armazena os arquivos de configuração específicos do sistema.
Com isso, a extração desses arquivos conterão as configurações de cada pacote e do
sistema. Outras configurações podem estar armazenadas em outros diretórios,
mas o Cupper ainda não provê funcionalidade para recupera-los. % TODO: Adicionar isso aos trabalhos futuros

Nem todos os arquivos disponíveis no \texttt{/etc/} estão relacionados a um pacote.
Arquivos postos nesse diretório de forma manual, ou seja, sem a utilização de pacotes,
podem estar presentes. Para evitar a coleta desnecessárias desses tipos de arquivos,
o Cupper não extraí arquivos que não estejam relacionados a um pacote.

Nas plataformas Debian \textit{based}, o plugin utiliza a ferramenta \texttt{dpkg}
para identificar os arquivos que estão relacionados com um pacote. Nas plataformas
Archlinux \textit{based}, o plugin utiliza a ferramenta \texttt{pacman} com o mesmo propósito.
O que difere a extração de uma plataforma para a outra é a interpretação da saída
de cada ferramenta.

As informações de \textit{owner},\textit{group} e \textit{mode} são extraídas com a ferramenta \texttt{ls},
o conteúdo com a ferramenta \texttt{cat} e o tipo com a ferramenta \texttt{file}
que estão presentes em todas as distribuições Linux.
O Código~\ref{code:files_json} mostra um exemplo do JSON gerado pelo plugin.

\noindent\begin{minipage}{\textwidth}
  \lstset{style=shell}
  \lstinputlisting[
    label=code:files_json,
    caption="Saída JSON do plugin Ohai \textit{files}"]{editaveis/code/json_files.json}
\end{minipage}\hfill

Os arquivos considerados sensíveis, ou seja, com conteúdo sigiloso, como
senhas, chaves ou certificados, podem ser especificados no arquivo de
configuração \texttt{.sensibles} no diretório raiz do projeto Cupper. Com isso,
esses arquivos não serão extraídos, sendo decisão do usuário a melhor forma
de recuperar esses arquivos.

\subsection{Bloco de Receita Gerado}

É utilizado o recurso Chef \textit{cookbook\_file} para cada arquivo extraído.
Os atributos usados na receita gerada são \textit{source}, \textit{owner},
\textit{group} e \textit{mode} como mostra o Código~\ref{code:files_recipe}.

\noindent\begin{minipage}{\textwidth}
  \lstset{style=shell}
  \lstinputlisting[
    label=code:files_recipe,
    caption="Exemplo de receita gerada pela extração de arquivos de configuração"]{editaveis/code/files_recipe.rb}
\end{minipage}\hfill

O conteúdo do arquivo coletado é salvo no diretório \texttt{files} do projeto Cupper
com o caminho descrito no atributo \textit{source}.

\subsection{Problemas Encontrados}

A primeira abordagem da extração dos arquivos de configuração foi a coleta de todos
os arquivos do diretório \texttt{etc}. Era evidente a necessidade de filtragem dos
arquivos, então definimos alguns critérios para eliminar aqueles arquivos que não são
necessários ou deixa-lo como opcional.

Como dito antes, os arquivos sensíveis são aqueles que contem
informações sigilosas ou que pode causar perda de alguma funcionalidade importante.
Durante os testes, o arquivo \textit{sudoers} continha falhas de sintaxe o que causou
quebra completa no sistema. Sendo que este e outros arquivos fazerem parte das
configurações do sistema, decidimos manter a extração deles, mas deixando uma opção
para o usuário em copia-los ou não para a receita. Esta opção foi adicionada como
atributo no Cupperfile (Código~\ref{code:sensiblefiles})

\noindent\begin{minipage}{\textwidth}
  \lstset{style=shell}
  \lstinputlisting[
    language=Ruby,
    label=code:sensiblefiles,
    caption="Definição do atributo de \textit{sensible files}"]{editaveis/code/sensiblefiles.rb}
\end{minipage}\hfill

Arquivos que não são vinculados a um pacote são descartados. Essa abordagem força
a padronização de ambientes que seguem a instalação e configuração por pacote.


\section{Funcionalidade de Extrair Links}
\label{sec:links}

\subsection{Atributos}
Para o recurso Chef de \texttt{link} foi definido os seguintes
atributos para a extração:

\begin{itemize}
  \item \textit{group}: define o grupo que pertence o link;
  \item \textit{owner}: dono do link, referente ao usuário no ambiente;
  \item \textit{mode}: permissões do link;
  \item \textit{to}: caminho para o arquivo real;
\end{itemize}

\subsection{\textit{Plugin} de Extração}

O mesmo plugin utilizado para recuperar arquivos de configuração descritos
na Seção~\ref{sec:files} também é usado para extrair \textit{links}. Isso é possível
pelo atributo \textit{type} coletado pelo plugin. Nele está contido, além do tipo do arquivo,
o caminho para o arquivo real. Os outros atributos são comuns entre arquivos e
\textit{links}, que são coletados e tratados da mesma forma. O JSON gerado segue a
mesma estrutura dos arquivos como mostra o Código~\ref{code:json_links}.

\noindent\begin{minipage}{\textwidth}
  \lstset{style=shell}
  \lstinputlisting[
    label=code:json_links,
    caption="Saída JSON do plugin Ohai \textit{files} com exemplo de link."]{editaveis/code/json_links.json}
\end{minipage}\hfill

\subsection{Bloco de Receita Gerado}

A receita gerada usa o recurso \texttt{links} (Código~\ref{code:links_recipe}) com os
atributos para cada \textit{link} gerado.

\noindent\begin{minipage}{\textwidth}
  \lstset{style=shell}
  \lstinputlisting[
    label=code:links_recipe,
    caption="Exemplo de receita gerada pela extração de links."]{editaveis/code/links_recipe.rb}
\end{minipage}\hfill

O atributo \textit{mode} é irrelevante neste caso. Todos os \textit{links} gerados tem a mesma permissão
777, pois o que determina a leitura, escrita ou execução do \textit{link} é o arquivo para o qual ele
direciona. Sendo assim, esse atributo é irrelevante e será removido em atualizações futuras.

\subsection{Problemas Encontrados}

Durante os testes de configuração dos novos ambientes, os \textit{links} continham caracteres
especias que não eram reconhecidos pelo \textit{encode} padrão do Ruby causando erro.
Esse tipo de texto requer um tratamento especial dentro do código, mas as tentativas
de correção desse problema não tiveram resultados positivos. A alternativa que encontramos
para contornar temporariamente esse problema é a remoção de qualquer link ou arquivo que
não estão nos formatos de \textit{encode} padrão suportados pela ferramenta.


\section{Funcionalidade de Extrair Usuários}
\label{sec:users}

\subsection{Atributos}

Para a ter a funcionalidade mínima esperada desse recurso, definimos que iríamos precisar
extrair os seguintes atributos:


\begin{itemize}
    {\itshape\item name}: Identifica nome do usuário;
    {\itshape\item uid}: Identificador do usuário;
    {\itshape\item home}: Identificar diretório \textit{home} desse usuário;
    {\itshape\item shell}: Identificar qual \textit{shell} o usuário usa ;
    {\itshape\item manage\_home}: Identificar se a \textit{home} deve ser criada;
\end{itemize}

\subsection{\textit{Plugin} de Extração}
Não foi necessário desenvolver um \textit{plugin} para extrair usuários pois o Ohai
já possui um \textit{plugin} nativo para isso. Ele provê atributos referentes
a informações de usuários, grupos, permissões entre outras.

Esse \textit{plugin} utiliza o módulo Ruby \textit{etc} que provê acesso a informações
do sistema operacional, diferentemente de alguns outros \textit{plugins} do Ohai
que executam comandos \textit{shell} diretamente na máquina. A saída é interpretada
pelo \textit{plugin} e transformada em um JSON construído com a estrutura
\texttt{``etc''=> \{ ``passwd''=> \{ ``root''=> \{ ``dir''=> ``/root'', ``gid''=> 0, ``uid''=> 0, ``shell''=> ``/bin/bash'', ``gecos''=> ``root'' \}} 
como mostrada no Código~\ref{code:json_user}

\noindent\begin{minipage}{\textwidth}
  \lstset{style=shell}
  \lstinputlisting[frame=single,
    label=code:json_user,
    caption=Saída JSON do \textit{plugin} Ohai \textit{passwd} para usuários]{editaveis/code/json_user.json}
\end{minipage}\hfill

\subsection{Bloco de Receita Gerado}

O JSON do Código~\ref{code:json_user} é usado para construir a receita responsável pela
criação dos usuários. O recurso Chef \textit{user} é usado para essa criação com
os atributos \textit{name}, \textit{uid}, \textit{home}, \textit{shell} e 
\textit{manage\_home}, como pode ser observado no Código~\ref{code:user_recipe}.

\noindent\begin{minipage}{\textwidth}
  \lstset{style=shell}
  \lstinputlisting[
    language=Ruby,
    frame=single,
    label=code:user_recipe,
    caption=Exemplo de receita gerada pela extração de usuários.]{editaveis/code/user_recipe.rb}
\end{minipage}\hfill

\subsection{Problemas Encontrados}
A receita não conseguia ser executada em um novo ambiente, ou quando conseguia
ser executada, perdia as referências entre os grupos e os usuários. Isso fazia
com que alguns usuários perdessem privilégios de super usuário, ou que não
conseguissem executar nenhum comando.

Para solucionar esse problema uma funcionalidade de extração de grupos, não prevista
no escopo inicial, teve que ser implementada. Sua descrição está na Seção~\ref{sec:groups}.

\section{Funcionalidade de Extrair Grupos}
\label{sec:groups}

\subsection{Atributos}

Para a ter a funcionalidade mínima esperada desse recurso, definimos que iríamos precisar
extrair os seguintes atributos:

\begin{itemize}
    {\itshape\item gid}: Identificador do grupo;
    {\itshape\item name}: Nome do grupo;
    {\itshape\item members}: Identificar membros do grupo;
\end{itemize}


\begin{itemize}
    {\itshape\item append}: Definir que os membros podem ser adicionados e removidos;
    {\itshape\item action}: Definir ação de criar o grupo;
\end{itemize}

\subsection{\textit{Plugin} de Extração}
Não foi necessário criar um \textit{plugin} para o Ohai que extraísse esses atributos
referentes à grupos já que o Ohai já possui um. O nome \textit{plugin} usado é 
\textit{passwd} e ele provê informações de grupos, usuários, permissões entre outras.

\subsection{Bloco de Receita Gerado}

Para a criação da receita que cria os grupos em um novo ambiente, foi definido
como seria o \textit{template} dos blocos de recurso de grupo na sintaxe das receitas Chef.
Um exemplo pode ser visto no Código~\ref{code:groupresource}.

\noindent\begin{minipage}{\textwidth}
  \lstset{style=shell}
  \lstinputlisting[language=Ruby, frame=single, label=code:groupresource, caption="Bloco de recurso de grupo Chef"]{editaveis/code/groupresource.rb}
\end{minipage}\hfill

Se uma receita com esse recurso for executada em um ambiente, o grupo \textit{www-data}
irá ser modificado com um novo membro \textit{maintenance}.

Para gerar então diversos desses blocos na receita final o \textit{template} do Código~\ref{code:grouptemplate} foi definido.

\noindent\begin{minipage}{\textwidth}
  \lstset{style=shell}
  \lstinputlisting[language=Ruby, frame=single, label=code:grouptemplate, caption="\textit{Template} para os blocos de recursos de grupo Chef"]{editaveis/code/grouptemplate.rb}
\end{minipage}\hfill

Dessa forma, como visto acima, para criar um bloco de receita de recursos Chef é
necessário saber qual vai ser o \textit{template} da escrita desse bloco e qual \textit{plugin}
o coletor vai utilizar para a extração.

\subsection{Problemas Encontrados}

Essa funcionalidade de extração de grupos não tinha sido prevista inicialmente.
Como citado na Seção~\ref{sec:users}, ela nasceu de um problema com a criação de usuários
de outra funcionalidade. Sem a organização desses usuários em grupos, o Chef não conseguia
executar a receita.

% mapear quais problemas bota aqui e quais bota em users mesmo

\section{Funcionalidade de Extrair Serviços}
\label{sec:services}

\subsection{Atributos}

Para a ter a funcionalidade mínima esperada desse recurso, definimos que iríamos precisar
extrair os seguintes atributos:


\begin{itemize}
    {\itshape\item provider}: Identificar qual \textit{Init System} está gerenciando
esse serviço;
    {\itshape\item service\_name}: Identificar o nome do serviço a
ser levantado;
    {\itshape\item action}: Identificar ação, padrão é \textit{restart};
\end{itemize}

\subsection{\textit{Plugin} de Extração}
Para a extração desses atributos referentes aos serviços, foi criado um \textit{plugin} do
Ohai que, a partir do comando \textit{systemctl} lista todos os serviços do ambiente
e extrai os atributos de \textit{service\_name}, \textit{provider}
e \textit{action}.

Esses atributos são coletados e para cada serviços extraído um objeto de serviço
é criado para ser iterado no \textit{template} posteriormente.

\subsection{Bloco de Receita Gerado}
Para a criação da receita que cria e executa esses serviços em outro ambiente, foi 
definido como seria o \textit{template} dos blocos de recurso de serviço na sintaxe das 
receitas Chef. Um exemplo pode ser visto no Código~\ref{code:serviceresource}.

\noindent\begin{minipage}{\textwidth}
  \lstset{style=shell}
  \lstinputlisting[language=Ruby, frame=single, label=code:serviceresource, caption="Bloco de recurso de serviço Chef"]{editaveis/code/serviceresource.rb}
\end{minipage}\hfill

Esse exemplo, se executado em um ambiente, diz para o
gerenciador de serviços (\textit{Init System}) dar \textit{start} no serviço do Tomcat.

Dessa forma, para gerar diversos desses blocos na receita final, temos o \textit{template}
do Código~\ref{code:servicetemplate}.

\noindent\begin{minipage}{\textwidth}
  \lstset{style=shell}
  \lstinputlisting[language=Ruby, frame=single, label=code:servicetemplate, caption="Template para os blocos de recursos de serviço Chef"]{editaveis/code/servicetemplate.rb}
\end{minipage}\hfill

Então para gerar um bloco de receita de recursos de serviço Chef é necessário
saber como vai ser o \textit{template} da escrita desse bloco e qual \textit{plugin} o
coletor vai utilizar para a extração.

\subsection{Problemas Encontrados}

A primeira abordagem utilizada para a extração dos atributos referentes a esse
recurso não previa a distinção entre serviços instalados e serviços que estão
ativos ou estão rodando. A extração ocorria sem nenhum problema, e todos os serviços
instalados eram extraídos. Mas ao executar a receita em um novo ambiente, diversos
serviços falhavam ao tentar se inicializar.

Serviços que já estavam falhando ou já estavam inativos no ambiente sendo extraído
também iriam falhar no ambiente novo a ser gerado, e isso teve que ser tratado.


Para solucionar esse problema só serviços que tem status \textit{loaded, active e running}
passaram a ser extraídos.\

Outro problema similar foi o de não verificar qual Init System gerenciava cada
serviço. Já que o recurso de serviço do Chef considera por padrão que os serviços
são \textit{systemd}, quando algum serviço não era, falhas aconteciam.

A extração passou então a extrair o atributo \textit{provider} que ao ser passado
para o recurso de serviço do Chef, solucionava esse problema.

O comando utilizado para que o \textit{plugin} do Ohai extraia os serviços não
lista somente serviços em si, já que ele lista todo tipo de \textit{systemd units}
como \textit{mounts, devices, targets, paths}. Esse comportamento não é ideal
tanto por gerar falhas quanto por não fazer parte do escopo interagir com esses
outros tipos de \textit{systemd units}. O \textit{plugin} passou a filtrar as
unidades e só pegar os que são serviços.

\section{Funcionalidade Gerar Diretório de Projeto}
\label{sec:proj}

Essa funcionalidade cria a estrutura de diretório para que o Cupper gere as 
receitas Chef e leia o seu arquivo de configuração, o Cupperfile.

Como exibido no Código~\ref{code:thorsaida2}, o básico dessa funcionalidade é, bem
simples. Um diretório com o nome escolhido vai ser criado, dentro dele um diretório
para as cookbooks, um arquivo oculto para listar arquivos sensíveis, e um 
Cupperfile para as definições que o usuário define para a extração.

\subsection{Definição de Ambiente}
Existem algumas classes importantes no código do Cupper que são relacionadas a
definição do ambiente de execução. Umas deas é a classe do \textit{Cupperfile} e outra
é o \textit{Environment}.

A classe de \textit{Environment} é responsável por checar se o ambiente é um
ambiente válido com a presença de um Cupperfile, definir o diretório do ambiente
(\textit{root\underline{ }path}) e instanciar um \textit{Cupperfile}. A classe \textit{Cupperfile} carrega as configurações definidas no arquivo Cupperfile.

\subsection{Problemas encontrados}
Devido a priorização realizada, e com o andamento do projeto, o carregamento de
configurações adicionais vindas do Cupperfile tinha sido tirado do backlog do
trabalho.

Essa decisão seguiu por algum tempo até que algumas falhas fizeram o Cupperfile
ser indispensável para deixar o Cupper funcionando para mais casos.

\subsubsection{Outras Abordagens de Leitura do Cupperfile}
Para a leitura e carregamento das configurações definidas no Cupperfile diversas
abordagens foram consideradas e testadas. A primeira abordagem foi espelhar a
leitura do Cupperfile ao que o Vagrant faz com o Vagrantfile. O Vagrant é outra
ferramenta dentro do contexto de DevOps que inspirou algumas das decisões arquiteturais
do Cupper. Assim como o Vagrant, essa abordagem iria carregar as configurações
do Cupperfile que seria escrito em Ruby. O código do Cupperfile seria carregado
utilizando o \textit{Kernel.load}, que carrega e executa códigos externos.
Essa abordagem foi abandonada pois previa diversas instancias do \textit{loader}
e de objetos que possuiriam informações de vários ambientes (como é necessário
no Vagrant), e isso se mostrou desnecessário e muito complexo para o Cupper.

A segunda abordagem para o carregamento das configurações do Cupperfile foi a de
utilizar uma Gem externa (\textit{onfiguration}) que define um padrão de escrita de 
arquivos de configuração (semelhante a um JSON), e permite o carregamento 
simples deles. Essa também foi abandonada por adicionar mais dependencias para o Cupper.

A terceira e quarta abordagens voltaram a tentar utilizar Ruby puro para resolver
esse processo de carregamento de arquivos de configuração, dessa vez tentando 
fazer um módulo mais simples que prevesse somente um ambiente em que o Cupper
iria operar. A quarta abordagem consistiu de somente tornar isso global e
pertencente do módulo \textit{Cupper::Config} acessível em todo o código.

\textit{Links} úteis para carregamento de arquivos de configuração de todas essas
abordagens podem ser encontrados no Apêndice na Seção~\ref{apc:conf-file}.

\subsection{Mudanças e Justificativas}
O escopo tinha sido alterado, e o carregamento de configurações do Cupperfile
tinha sido removido do backlog. Mas esse carregamento se mostrou necessário
e acabou voltando para o desenvolvimento.

Algumas dessas definições do Cupperfile foram reativas, e implementadas de acordo
com o surgimento da necessidade de implementar. Outras tinham sido previstas, mas
não prioritárias, como a de \textit{allow downgrade}.

O que fez essa abordagem voltar a ser indispensável foi o fato de que o Cupper
não pode gerar uma receita que quebra o acesso do usuário a um sistema, ou pior,
que inutilize o sistema.

\section{Funcionalidade de Listar \textit{Plugins}}
\label{sec:list}

Listar \textit{plugins} é uma funcionalidade de suporte para o usuário
que mostra quais os plugins adicionais do Ohai que estão
disponíveis. É importante realizar a verificação, pois
sem a disponibilidade deles não é possível realizar a extração
dos atributos do ambiente.

\section{Funcionalidade de Ajuda}
\label{sec:help}

A funcionalidade de ajuda trata basicamente do comando que imprime na tela um resumo
dos outros comandos e suas opções. Já que o Cupper segue os padrões de ferramentas
CLI (\textit{Command Line Interface}), ter um comando de ajuda se faz necessário
e útil.

Para montar a estrutura de comandos foi utilizado o Thor, uma Gem para criação
de CLIs. Ele facilita o \textit{parser} de entradas e a documentação de cada comando criado.
O próprio Thor facilita a criação dos comandos de ajuda pra cada um dos comandos
criados. Basta documentar cada método que será um comando com \textit{templates} de cada
comando e suas descrições como na primeira linha do exemplo do Código~\ref{code:thor}

\noindent\begin{minipage}{\textwidth}
  \lstset{style=shell}
  \lstinputlisting[language=Ruby, frame=single, label=code:thor, caption="Exemplo de descrição para documentação e ajuda"]{editaveis/code/thor.rb}
\end{minipage}\hfill

\subsection{Ajuda Geral}
Utilizando das funcionalidades do Thor, a ajuda geral imprime a lista de todos
os comandos base, como utilizá-los e suas descrições. O que o Thor faz é basicamente
imprimir todos os \textit{desc}`s presentes nós métodos de comando.

Um exemplo de saída pode ser visto no Código~\ref{code:thorsaida}.

\noindent\begin{minipage}{\textwidth}
  \lstset{style=shell}
  \lstinputlisting[language=Bash, frame=single, label=code:thorsaida, caption="Exemplo de saída de ajuda geral do Thor"]{editaveis/code/thorsaida.sh}
\end{minipage}\hfill


\subsection{Ajuda Específica}
Semelhante a ajuda geral, mas para um comando específico. O Thor somente imprime
na tela o \textit{desc} de um comando específico como pode ser visto no Código~\ref{code:thorsaida2}.

\noindent\begin{minipage}{\textwidth}
  \lstset{style=shell}
  \lstinputlisting[language=Bash, frame=single, label=code:thorsaida2, caption="Exemplo de saída de ajuda específica do Thor"]{editaveis/code/thorsaida2.sh}
\end{minipage}\hfill

\section{Funcionalidade Não Implementadas}
\label{sec:rel}

\subsection{Funcionalidade de Gerar \textit{Templates}}

\subsection{Funcionalidade Gerar Relatório Intermediário}
Essa funcionalidade trata da geração de um relatório intermediário antes da
receita em si, para diagnóstico do sistema sendo extraído e para \textit{debug} do
Cupper. Ela prevê extrair informações adicionais de hardware, de sistema, e de
diversos aspectos como descrito na Seção~\ref{sec:cam-amb}. Essa funcionalidade
não foi implementada pois não houve tempo hábil para tal.

\subsection{\textit{Plugins} de Extração}
Para a extração dos atributos de cada camada descrita na Seção~\ref{sec:cam-amb},
diversos \textit{plugins} devem ser implementados, e vários \textit{plugins} já prontos do Ohai
devem ser utilizados.

A necessidade ou não da implementação, e quais comandos acessam a informação
que se deseja podem ser vistos nas Tabelas~\ref{tab:atrhard},~\ref{tab:hdmount},~\ref{tab:so},~\ref{tab:app},~\ref{tab:config} e~\ref{tab:service}.

\subsection{Geração do Relatório}
A saída do relatório será em um formato JSON (\textit{JavaScript Object Notation}),
assim como a saída oficial da Gem do Ohai quando se executa em um ambiente. Será,
muito semelhante a execução da Gem do Ohai, mas além de utilizar os \textit{plugins} nativos
do Ohai, também utilizará todos os \textit{plugins} custom do Cupper e do Ohai.

\subsection{Mudanças e Justificativas}
Essa funcionalidade não foi implementada devido a priorização realizada e andamento
do projeto. Funcionalidades que traziam mais valor para o usuário, como 
gerar a receita com os recursos Chef em si tiveram mais prioridade e foram implementadas
primeiro.

