\section{Coleta e Análise de Resultados}

Nesta Seção será apresentado o método de coleta e análise dos resultados deste % TODO: melhorar
trabalho para validação da proposta. Como disposto nos objetivos (\ref{sec:obj}),
o resultado da execução do Cupper é um \textit{script} em formato de receita Chef.
Tal receita poderá ser utilizada pelo Chef para replicar o ambiente. Sendo assim
o foco da coleta e análise de dados está na capacidade de replicar um ambiente
utilizando o Cupper e Chef de maneira automatizada.

Divide-se a coleta e análise de dados em quatro etapas:

\textbf{Etapa 1}: Nesta etapa serão coletados os dados sobre quais informações o Cupper consegue
extrair e quais são as replicações. A coleta será feita por um \textit{checklist} que contenha
todos os itens propostos para implementação da ferramenta. O \textit{checklist} será
utilizado para delimitar os dados base que serão coletados do ambiente para
a validação. Também é um informativo sobre até qual ponto o projeto conseguiu
alcançar.

\textbf{Etapa 2}: Nesta etapa será construído um ambiente que seja possível extrair as configurações
sem a utilização do Cupper. Essas informações estarão alinhadas ao \textit{checklist} da etapa
anterior e serão a base para o comparativo com o ambiente replicado.

\textbf{Etapa 3}: Nesta etapa a ferramenta Cupper irá ser executada no ambiente construído na etapa
anterior. As receitas geradas serão usadas pelo Chef em um ambiente limpo que contenha
apenas as configurações mínimas para o seu funcionamento, ou seja, um sistema
operacional (Debian ou Arch), acesso por interface de rede e o Chef (a Seção
~\ref{sec:chef} define o ambiente mínimo para o funcionamento do Chef).

\textbf{Etapa 4}: Nesta etapa é feito a extração sem utilização do Cupper, assim como foi realizado
na etapa 2, no ambiente replicado. Então será feito uma comparação das configurações
dos dois ambientes.

A conclusão dos dados coletado e analizados será realizado pela porcentagem de
item correspondendes dos ambientes como demonstrado na Tabela \ref{tab:ex_result}.

\begin{table}[H]
  \centering
  \caption{Exemplo de tabela de comparação dos resultados.}
  \label{tab:ex_result}
  \setlength\extrarowheight{5pt}
  \begin{tabular}{c|c|c|c}
    \hline
    \rowcolor[HTML]{EFEFEF} 
  \textbf{Camada}                      & \textbf{\begin{tabular}[c]{@{}c@{}}Ambiente\\ (Extração Manual)\end{tabular}} & \textbf{\begin{tabular}[c]{@{}c@{}}Ambiente Replicado\\ (Extração Manual)\end{tabular}} & \multicolumn{1}{l|}{\cellcolor[HTML]{EFEFEF}\textbf{Replicado}} \\ \hline
                                         & Pacote Nginx Instalado                                                        & Pacote Nginx Instalado                                                                  & X                                                               \\ \cline{2-4} 
                                         & Configuração Nginx Aplicada                                                   & Configuração Nginx Aplicada                                                             & X                                                               \\ \cline{2-4} 
    \multirow{-3}{*}{Aplicação} & Pacote PostgreSQL Instalado                                                   & Pacote PostgreSQL Instalado                                                             & X                                                               \\ \hline
    Serviço                     & Serviço Nginx Executando                                                      & Serviço Nginx Não Executando                                                            &                                                                 \\ \hline
    \multicolumn{3}{c|}{Porcentagem Replicada}                                                                                                                                                                    & 75\%                                                            \\ \hline
  \end{tabular}
\end{table}

