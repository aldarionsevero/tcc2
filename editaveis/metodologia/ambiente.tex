\section{Análise do Ambiente}

Neste trabalho, é considerado como ambiente uma máquina composta
de hardware e sistema operacional. São consideradas máquinas físicas
e virtuais. Essa Seção irá descrever como será a análise e extração
de informações do ambiente.

\subsection{Definição de Camadas}

Para decidir até que ponto a aplicação irá analisar as configurações
do ambiente é preciso definir os tipos de configuração, as características para
cada, e separar esses tipos em camadas onde a aplicação pode atuar.
Com essas camadas definidas, identifica-se quais camadas são relevantes e viáveis
para o escopo do trabalho.

%TODO: adicionar como foi feita a definição das camadas

\subsection{Definição de Critérios para a Seleção}
\label{sec:defcritcamada}
Após levantar as camadas de configuração do ambiente, é necessário definir em
quais camadas e em quais dos seus atributos o Cupper realmente vai atuar. 
Os critérios vão estar relacionados a dois aspectos importantes: a \textbf{relevância} 
para o projeto e a quantidade de \textbf{esforço} e \textbf{tempo} para a implementação durante
o Trabalho de Conclusão de Curso.

Para classificar um atributo como relevante para análise, ele deve seguir os
seguintes critérios:

\begin{enumerate}
\item Um atributo é relevante quando ele é necessário para a geração de uma
receita Chef que atenda aos nossos requisitos. 

Na Seção~\ref{sec:lev-rec} e~\ref{sec:escopo} definimos os recursos Chef
que iremos utilizar nas criações de \textit{cookbooks} e até que ponto 
avançaremos nas funcionalidades do Cupper, e dessa forma, os atributos
são relevantes se eles se enquadram nesse patamar, para alcançar esses objetivos.

Um exemplo básico disso é a arquitetura da CPU, que pode alterar a instalação
e qual a versão de pacotes a serem instalados.

\item Um atributo também é relevante se é necessário para criação de \textit{logs} 
e \textit{debugs}, tanto para uso da ferramenta durante o processo de gerar 
\textit{cookbooks} quanto para algum retorno para o usuário, para ajudar 
a entender possíveis erros ao executar
o Cupper.
\end{enumerate}

Com relação à dificuldade de implementação e de integração ao Cupper o
atributo deve seguir os seguinte critérios em ordem de prioridade:

\begin{enumerate}
\item O atributo tem maior prioridade de ser selecionado se é um dos dados 
que o Ohai lança por padrão e sem extensões.
\item O atributo tem média prioridade se, não é um dos dados que o Ohai
lança por padrão e sem extensões, mas existe um comando do sistema que facilita 
sua extração, e dessa forma facilitando a criação de \textit{plugins} pro 
Ohai.
\item O atributo tem menor prioridade se não é lançado pelo Ohai e nem tem 
comandos do sistema para recuperá-lo.
\end{enumerate}

