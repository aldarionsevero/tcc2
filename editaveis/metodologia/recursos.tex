\section{Definição dos Recursos Chef}
\label{sec:def_recurso}

A ferramenta Chef provê recursos para total automação da infraestrutura.
Sendo possível preparar, configurar e integrar a infraestrutura com a
flexibilidade de manutenção de scripts \cite{sharma:2015}. Para isso, o
Chef tem um estrutura complexa envolvendo diversos componentes para
o completo funcionamento e utilização de todos os recursos.

De modo análogo ao levantamento de camadas de ambiente, os recursos
de \textit{cookbooks} e do Chef necessários para implementação precisam
ser levantados e então selecionados para o escopo do projeto.

O principal objeto utilizado para a pesquisa do levantamento dos recursos
é a documentação oficial do Chef. A documentação é extensa e completa e
com base nela será feito um levantamento para avaliar quais recursos
serão necessários para a implementação deste projeto. Além disso, existem
referências não oficiais de publicações sobre a ferramenta abordando os
mesmos recursos.

\subsection{Definição de Critérios de Seleção de Recursos}
\label{sec:defcritrecurso}

De maneira geral, se um recurso cabe em uma das funcionalidades do escopo da
implementação, ele será selecionado para esse escopo. O escopo, e as
funcionalidades podem ser encontrados na Seção~\ref{sec:escopo}. É possível
generalizar e considerar que recursos que não se enquadram na categoria de
\textit{Custom} serão selecionados.
