\chapter{Conclusão}
\label{chap:conclusao}

A proposta deste trabalho é desenvolver uma ferramenta que extraia, de maneira
automatizada, as informações de uma máquina e construa \textit{script} em padrão
de receita Chef. Baseando-se nos experimentos realizados nos ambientes de teste propostos,
concluímos que o Cupper tem sua funcionalidade mínima esperada, conseguindo
extrair os principais atributos de cada camada para reproduzir isso em um novo ambiente.

Algumas limitações estão presentes na versão atual. Correções manuais podem ser necessárias
a depender das configurações extraídas, como pacotes fora de ordem ou sem repositório fonte.
É necessário adotar algumas premissas para o ambiente a ser replicado: ser possível executar
o Chef, repositórios de pacotes devem estar de acordo com o antigo ambiente e a distro atual
suportada e testada deve ser o Debian.

\section{Escopo Final do Projeto}

O projeto seguiu realizando replanejamentos e priorizações para alcançar o seu estado atual.
A realease referente ao trabalho aqui escrito abrangeu as funcionalidades de gerar 
\textit{cookbook} que engloba extração de pacotes instalados; de arquivos de configuração 
localizados no diretório \textit{/etc}; de \textit{links} localizados nesse mesmo diretório; 
de usuários e grupos e de serviços executados e ativos. Além disso, abrangeu as 
funcionalidades de CLI de criação do direorio padrão do Cupper com um Cupperfile, de listagem
de \textit{plugins} e de impressão de texto de ajuda.

\section{Trabalhos Futuros}

Além das funcionalidades que ainda precisam ser implementadas descritas na Seção~\ref{sec:rel},
há também a possibilidade de evolução da ferramenta para extrair e reproduzir
infraestruturas mais complexas. Um exemplo seria funcionalidades que
prevejam mais customização por parte do usuário, com mais atributos previstos no Cupperfile.
Outro exemplo seria a possibilidade de extrair configurações de vários ambientes simultaneamente
por meio da rede, e gerar receitas que possam reproduzir esse
conjunto de ambientes em uma nova rede, por exemplo com a utilização da ferramente Chake (\url{https://gitlab.com/terceiro/chake}).

Os trabalhos futuros estão listados em \textit{issues} no repositório oficial do Cupper.
Ao receber mais contribuições de novos desenvolvedores, e ir a cada \textit{release} se tornando
uma ferramente que segue tendências DevOps, o Cupper pode se tornar parte da \textit{stack} de ferramentas
que equipes utilizam, e futuramente entrar na família de ferramentas Chef.
