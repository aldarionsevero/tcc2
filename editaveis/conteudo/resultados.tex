
\chapter{Resultados e Discussão} % (fold)
\label{cha:analise_de_resultados}

Neste capitulo serão analisados os resultados obtidos no capitulo anterior, que servirão de motivo para a tomada de decisão de como a contribuição será realizada. Resultados como desempenho, consumo de memória e custos de implementação serão apontadas com o objetivo de comparação entre os algoritmos.

\section{Contribuições} % (fold)
\label{sec:contribui_es}

Para a realização deste trabalho foram feitos contatos com os desenvolvedores oficiais do {\code APT} por duas vias distintas. Na primeira contribuição, o \textit{pull request} foi comprometido devido ao \textit{Travis CI} apontar a quebra de três testes que já estavam no repositório. A primeira forma de comunicação então foi via email com o responsável por essa contribuição, explicando que havia uma submissão, porém ela provavelmente seria ignorada devido ao teste falho. A resposta do email explicava que alguns testes de integração do {\code APT} realizam concorrência e podem vir a falhar, assim como um teste especifico que tentava baixar o mesmo pacote duas vezes para comparar seu tamanho, porém o \textit{Travis CI} eventualmente limitava a banda da seção, o que resultava na falha do teste por perca de conexão. Pouco depois já haviam os primeiros comentários sobre o primeiro \textit{pull request} pelo GitHub. Porém ao final dos comentários, sugestões e correções do deste \textit{pull request}, houve um pedido de confirmação ao dono do repositório se o comentário no código original apontando um pedido de correção ainda tinha necessidade e se a submissão daquela contribuição poderia ser aprovada. Infelizmente, ate a data de entrega deste trabalho não havia recebido resposta para essa pergunta do dono do repositório.

Como consequência, há um \textit{pull request} aberto com três \textit{commits} esperando uma revisão contendo as três contribuições programadas para este trabalho. Ate a entrega deste trabalho, não havia qualquer comentário nas contribuições.

% section contribui_es (end)


\section{\nmu Algoritmos de \textit{string matching} exatos} % (fold)
\label{sec:algor_timos_de_string_matching_exatos}

Para a busca com algoritmos de \textit{string matching} exatos, foram considerado três abordagens distintas:

\begin{description}
	\item[Expressão Regular:] Método atual de verificação. Foi comentado previamente  na \autoref{sec:algoritimos_de_textit}.
	\item[Knuth-Morris-Pratt:] Algoritmo que faz de uso do conceito de autômatos de estados para acelerar a busca. Previamente apresentado na \autoref{ssub:knuth_morris_pratt_}.
	\item[Rabin-Karp:]  Algoritmo que faz de uso de \textit{hash}. Ver \autoref{ssub:rabin_karp}.
\end{description}

Após a coleta de $150$ execuções de busca para dez pacotes usando os algoritmos apontados acima, a \autoref{ssub:rabin_karp} foi elaborada a partir da a mediana dos tempos destas execuções.

\begin{figure}[htbp]
  \centering
  \includegraphics[width=0.95\textwidth]{figuras/tempo-rk_kmp_std}
  \caption{Estimativa de tempo para pacotes usando algoritmos de busca exata}
  \label{tempo_rk_kmp_std}
\end{figure}

Como podemos observar, tanto o \textit{Rabin-Karp} quanto o \textit{KMP} tiveram um tempo de execução de cerca de $\frac{1}{3}$ do tempo gasto atualmente com o uso de expressões regulares, sendo que, no geral, o algoritmo de \textit{Rabin-Karp} possui um desempenho ligeiramente melhor.


\begin{figure}[htbp]
  \centering
  \includegraphics[width=0.95\textwidth]{figuras/memory_rk.png}
  \caption{Uso de memória com uso do algoritmo de \textit{Rabin-Karp}}
  \label{memory_rk}
\end{figure}

O consumo total de memória para ambos os algoritmos, \textit{Rabin-Karp} na \autoref{memory_rk} e expressões regulares na \autoref{memory_std},  apresentaram resultados similares, porém o método de \textit{Rabin-Karp} faz um uso de memória mais pontual, alcançando o pico de consumo ao final do processo, quando esta prestes a liberar os recursos. Já o método de expressões regulares apresenta um consumo aproximadamente linear de memória. 


\begin{figure}[htbp]
  \centering
  \includegraphics[width=0.95\textwidth]{figuras/memory_regex.png}
  \caption{Uso de memória das expressões regulares}
  \label{memory_std}
\end{figure}

Um gasto mais prologado de memória vem a ser prejudicial para sistemas em que diversos processos possam estar sendo executados em conjunto; Todavia, o grau de consumo é baixo, evitando que este consumo por um período mais extenso venha a ser prejudicial. Em um sistema com 2GB de memória RAM, os 15MB utilizados representam menos de $1\%$ do total dos recursos disponíveis.
% chapter analise_de_resultados (end)


\section{Algoritmos de \textit{string matching} inexatos} % (fold)
\label{sec:algor_timos_de_string_matching_inexatos}

Para a busca com algoritmos de \textit{string matching} inexatos, foram considerado dois métodos distintos:

\begin{description}
	\item[Levenshtein:] Método voltado para correção de erros. Foi previamente comentado na \autoref{sec:leveinstein}.
	\item[Coeficiente de Sørensen–Dice:] Método que faz de uso do conceito de autômatos de estados para agilizar a busca. Previamente apresentado na \autoref{ssub:s_rensen_dice_coefficient}.
\end{description}

Para analise de desempenho, foram coletadas 150 amostras de tempo para 10 buscas por pacotes distintos das quais metade eram de pacotes que não existiam, semelhante realizado para os \lnameref{sec:algor_timos_de_string_matching_exatos}. A \autoref{tempo_rk_kmp_std_lev_dic} retrata a mediana destes tempos.

\begin{figure}[htbp]
  \centering
  \includegraphics[width=0.95\textwidth]{figuras/tempo-rk_kmp_std_lev_dice}
  \caption{Estimativa de tempo para pacotes usando algoritmos de busca inexata}
  \label{tempo_rk_kmp_std_lev_dic}
\end{figure}

Como apresentado na primeira etapa deste trabalho\footnote{Trabalho de conclusão de curso 1: Algoritmo para Qualificação das Saídas de Buscas em
Gerenciadores de Repositórios de Distribuições Linux}, o algoritmo de \textit{Leveinstein} apresenta um dos melhores tempos para buscas inexatas de \textit{strings}. A \autoref{tempo_rk_kmp_std_lev_dic} reforça essa afirmação, ao mostrar que o tempo de resposta deste algoritmo é inferior ao tempo gasto para buscas com expressões regulares e ficando pouco acima do tempo necessário para realizar buscas exatas com o algoritmo de  \textit{Rabin-Karp}.

% \section{Resultados Gerais} % (fold)
% \label{sec:resultados_gerais}

% subsection tempo (end)
% section resultados_gerais (end)
\chapter{Considerações Finais} % (fold)
\label{cha:dificuldades_encontradas}
\section*{Trabalhos Futuros} % (fold)
\label{sec:trabalhos_futuros}


% section trabalhos_futuros (end)

