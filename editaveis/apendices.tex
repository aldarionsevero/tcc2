\begin{apendicesenv}

\partapendices

\chapter[Tabelas de cada camada]{Tabelas representando atributos de cada camada do ambiente}
\label{apc:tabelas}

\begin{table}[H]
\centering
\caption{Atributos Relacionados a CPU, Memória e \textit{Motherboard}}
\label{tab:atrhard}
\begin{tabular}{c|c|c|c|c|c}
\hline
\rowcolor[HTML]{C0C0C0} 
Atributo                                                        & Origem    & Relevante? & Ohai & Viável? & Selecionado? \\ \hline
Modelo CPU                                                      & lscpu     & não        & sim  & sim     & não          \\ \hline
\begin{tabular}[c]{@{}c@{}}Modo de \\ Operação CPU\end{tabular} & lscpu     & sim        & sim  & sim     & sim          \\ \hline
Arquitetura CPU                                                 & lscpu     & sim        & sim  & sim     & sim          \\ \hline
Cores                                                           & lscpu     & sim        & sim  & sim     & sim          \\ \hline
Stepping CPU                                                    & lscpu     & não        & sim  & sim     & não          \\ \hline
Frequência CPU                                                  & lscpu     & sim        & sim  & sim     & sim          \\ \hline
Cache CPU                                                       & lscpu     & sim        & sim  & sim     & sim          \\ \hline
Flags CPU                                                       & lscpu     & sim        & sim  & sim     & sim          \\ \hline
\begin{tabular}[c]{@{}c@{}}Modelo \\ Motherboard\end{tabular}   & dmidecode & não        & não  & não     & não          \\ \hline
Versão BIOS                                                     & dmidecode & não        & não  & não     & não          \\ \hline
Flags BIOS                                                      & dmidecode & não        & não  & não     & não          \\ \hline
Slots Memória                                                   & dmidecode & não        & sim  & sim     & não          \\ \hline
Marca Memória                                                   & dmidecode & não        & não  & não     & não          \\ \hline
Tamanho Memória                                                 & free      & sim        & sim  & sim     & sim          \\ \hline
Clock Memória                                                   & dmidecode & não        & não  & não     & não          \\ \hline
Total Swap                                                      & free      & sim        & sim  & sim     & sim          \\ \hline
Livre Swap                                                      & free      & sim        & sim  & sim     & sim          \\ \hline
Total Memória                                                   & free      & sim        & sim  & sim     & sim          \\ \hline
Livre Memória                                                   & free      & sim        & sim  & sim     & sim          \\ \hline
Buffers Memóra                                                  & free      & sim        & sim  & sim     & sim          \\ \hline
Cached Memória                                                  & free      & não        & sim  & sim     & não          \\ \hline
Ativo Memória                                                   & free      & sim        & sim  & sim     & sim          \\ \hline
Inativo Memória                                                 & free      & sim        & sim  & sim     & sim          \\ \hline
Sujo Memória                                                    & pmap      & sim        & sim  & sim     & sim          \\ \hline
\begin{tabular}[c]{@{}c@{}}Pag. Anon\\ Memória\end{tabular}     & pmap      & não        & sim  & sim     & não          \\ \hline
\end{tabular}
\end{table}

\begin{table}[H]
\centering
\caption{Atributos relacionados a Partições, Armazenamento, USB, PCI, e Rede}
\label{tab:hdmount}
\begin{tabular}{c|c|c|c|c|l}
\hline
\rowcolor[HTML]{C0C0C0} 
Atributo                                                                 & Origem                                                                & Relevante? & Ohai    & Viável & Selecionado? \\ \hline
\begin{tabular}[c]{@{}c@{}}Nome\\ dos\\ Dispositivos\end{tabular}        & lsblk                                                                 & sim        & sim     & sim    & sim         \\ \hline
\begin{tabular}[c]{@{}c@{}}Major/Minor\\ Dispositivos\end{tabular}       & lsblk                                                                 & sim        & não     & sim    & sim         \\ \hline
Removíveis                                                               & lsblk                                                                 & não        & não     & não    & não         \\ \hline
\begin{tabular}[c]{@{}c@{}}Somente\\ Leitura\end{tabular}                & lsblk                                                                 & não        & não     & não    & não         \\ \hline
Partições                                                                & lsblk                                                                 & sim        & sim     & sim    & sim         \\ \hline
\begin{tabular}[c]{@{}c@{}}Tamanho\\ Partições/\\ Discos\end{tabular}    & \begin{tabular}[c]{@{}c@{}}lsblk/\\ df\end{tabular}                   & sim        & sim     & sim    & sim         \\ \hline
\begin{tabular}[c]{@{}c@{}}Tipo\\ Filesystem\\ Dispositivos\end{tabular} & lsblk                                                                 & sim        & sim     & sim    & sim         \\ \hline
\begin{tabular}[c]{@{}c@{}}UUID\\ Dispositivos\end{tabular}              & lsblk                                                                 & sim        & sim     & sim    & sim         \\ \hline
Mountpoints                                                              & mount                                                                 & sim        & sim     & sim    & sim         \\ \hline
\begin{tabular}[c]{@{}c@{}}Dsipositivos\\ USB\end{tabular}               & lsusb                                                                 & não        & não     & não    & não         \\ \hline
\begin{tabular}[c]{@{}c@{}}Dsipositivos\\ PCI\end{tabular}               & lspci                                                                 & sim        & não     & sim    & sim         \\ \hline
\begin{tabular}[c]{@{}c@{}}Dispositivos \\ de Rede\end{tabular}          & \begin{tabular}[c]{@{}c@{}}lspci/\\ ifconfig/\\ dmidecode\end{tabular} & sim        & sim     & sim    & não         \\ \hline
\end{tabular}
\end{table}
%\input{artefatos/visao}
\end{apendicesenv}

\begin{table}[H]
\centering
\caption[Atributos do Sistema]{Atributos relacionados ao Sistema Operacional, Kernel e configs gerais do sistema}
\label{tab:so}
\begin{tabular}{c|c|c|c|l|c}
\hline
\rowcolor[HTML]{C0C0C0} 
Atributo                                                       & Origem               & Relevante? & Ohai & Viável? & Selecionado? \\ \hline
OS                                                             & /etc/distro-release  & sim        & sim  & sim     & sim         \\ \hline
Kernel                                                         & uname                & sim        & sim  & sim     & sim         \\ \hline
\begin{tabular}[c]{@{}c@{}}Versão\\ Kernel\end{tabular}        & uname                & sim        & sim  & sim     & sim         \\ \hline
\begin{tabular}[c]{@{}c@{}}Machine\\ Kernel\end{tabular}       & uname                & sim        & sim  & sim     & sim         \\ \hline
\begin{tabular}[c]{@{}c@{}}Módulos\\ Kernel\end{tabular}       & lsmod                & sim        & sim  & sim     & sim         \\ \hline
Distribuição                                                   & lsb\_release         & sim        & sim  & sim     & sim         \\ \hline
\begin{tabular}[c]{@{}c@{}}Release\\ Distribuição\end{tabular} & lsb\_release         & sim        & sim  & sim     & sim         \\ \hline
\begin{tabular}[c]{@{}c@{}}Família\\ Distribuição\end{tabular} & lsb\_release         & sim        & sim  & sim     & sim         \\ \hline
\begin{tabular}[c]{@{}c@{}}Geren.\\ de Pacotes\end{tabular}    & N/A                  & sim        & não  & sim     & sim         \\ \hline
\begin{tabular}[c]{@{}c@{}}Init \\ System\end{tabular}         & N/A                  & sim        & não  & sim     & sim         \\ \hline
\begin{tabular}[c]{@{}c@{}}Módulo \\ de Segurança\end{tabular} & N/A                  & sim        & não  & sim     & sim         \\ \hline
Hostname                                                       & hostname             & sim        & sim  & sim     & sim         \\ \hline
\begin{tabular}[c]{@{}c@{}}Interfaces\\ de Rede\end{tabular}   & ip                   & sim        & sim  & sim     & sim         \\ \hline
\begin{tabular}[c]{@{}c@{}}Endereços\\ de Rede\end{tabular}    & ip                   & sim        & sim  & sim     & sim         \\ \hline
\begin{tabular}[c]{@{}c@{}}Endereços\\ MAC\end{tabular}        & ip                   & sim        & sim  & sim     & sim         \\ \hline
\end{tabular}
\end{table}

\begin{table}[H]
\centering
\caption{Atributos relacionados às Aplicações}
\label{tab:app}
\begin{tabular}{c|c|c|c|c|c}
    \hline
    \rowcolor[HTML]{C0C0C0} 
    Atributo                                                         & Origem                                                 & Relevante? & Ohai & Viável & Selecionado? \\ \hline
Pacotes                                                              & \begin{tabular}[c]{@{}c@{}}pacman/\\ dpkg\end{tabular} & sim        & não  & sim    & sim          \\ \hline
\begin{tabular}[c]{@{}c@{}}Versões\\ Pacotes\end{tabular}            & \begin{tabular}[c]{@{}c@{}}pacman/\\ dpkg\end{tabular} & sim        & não  & sim    & sim          \\ \hline
Dependências                                                         & \begin{tabular}[c]{@{}c@{}}pacman/\\ dpkg\end{tabular} & sim        & não  & sim    & sim          \\ \hline
Arch                                                                 & \begin{tabular}[c]{@{}c@{}}pacman/\\ dpkg\end{tabular} & sim        & não  & sim    & sim          \\ \hline
    Pacotes Pip                                                      & pip                                                    & sim        & não  & sim    & sim          \\ \hline
\begin{tabular}[c]{@{}c@{}}Versões\\ Pacotes Pip\end{tabular}        & pip                                                    & sim        & não  & sim    & sim          \\ \hline
    Pacotes Ruby                                                     & gem                                                    & sim        & não  & sim    & sim          \\ \hline
\begin{tabular}[c]{@{}c@{}}Versões\\ Pacotes Ruby\end{tabular}       & gem                                                    & sim        & não  & sim    & sim          \\ \hline
\begin{tabular}[c]{@{}c@{}}Pacotes por\\ outro Ger.\end{tabular}     & N/A                                                    & não        & não  & não    & não          \\ \hline
\begin{tabular}[c]{@{}c@{}}Instalação\\ Manual\end{tabular}          & N/A                                                    & não        & não  & não    & não          \\ \hline
\end{tabular}
\end{table}

\begin{table}[]
\centering
\caption{Atributos relacionados às Configurações de Aplicações}
\label{tab:config}
\begin{tabular}{c|c|c|c|c|c}
\hline
\rowcolor[HTML]{C0C0C0} 
Atributo                                                & Origem                                                          & Relevante? & Ohai & Viável & Selecionado? \\ \hline
\begin{tabular}[c]{@{}c@{}}Host\\ Config\end{tabular}   & /etc                                                            & sim        & não  & sim    & sim          \\ \hline
\begin{tabular}[c]{@{}c@{}}Add-on\\ Config\end{tabular} & /etc/opt                                                        & sim        & não  & sim    & sim          \\ \hline
\begin{tabular}[c]{@{}c@{}}Sgml\\ Config\end{tabular}   & /etc/sgml                                                       & sim        & não  & sim    & sim          \\ \hline
\begin{tabular}[c]{@{}c@{}}X11\\ Config\end{tabular}    & /etc/X11                                                        & sim        & não  & sim    & sim          \\ \hline
\begin{tabular}[c]{@{}c@{}}Xml\\ Config\end{tabular}    & /etc/xml                                                        & sim        & não  & sim    & sim          \\ \hline
\begin{tabular}[c]{@{}c@{}}User\\ Config\end{tabular}   & \begin{tabular}[c]{@{}c@{}}/home/user/\\ .config\end{tabular}   & sim        & não  & sim    & sim          \\ \hline
\begin{tabular}[c]{@{}c@{}}User\\ Custom\end{tabular}   & \begin{tabular}[c]{@{}c@{}}/home/user/\\ .some-app\end{tabular} & não        & não  & não    & não          \\ \hline
\begin{tabular}[c]{@{}c@{}}Root\\ config\end{tabular}   & /root/*                                                         & não        & não  & não    & não          \\ \hline
\end{tabular}
\end{table}

\begin{table}[H]
\centering
\caption{Relação entre Distribuições e \textit{Init Systems}}
\label{tab:inits-distro}
\begin{tabular}{|c|cccccl}
\hline
\rowcolor[HTML]{C0C0C0} 
Upstart & \multicolumn{1}{c|}{\cellcolor[HTML]{C0C0C0}Systemd} & \multicolumn{1}{c|}{\cellcolor[HTML]{C0C0C0}SMF} & \multicolumn{1}{c|}{\cellcolor[HTML]{C0C0C0}SysV} & \multicolumn{1}{c|}{\cellcolor[HTML]{C0C0C0}OpenRC} & \multicolumn{1}{c|}{\cellcolor[HTML]{C0C0C0}RC} & \multicolumn{1}{l|}{\cellcolor[HTML]{C0C0C0}Launchd} \\ \hline
Ubuntu  & \multicolumn{1}{c|}{Fedora15}                        & \multicolumn{1}{c|}{Solaris}                     & \multicolumn{1}{c|}{RHEL5}                        & \multicolumn{1}{c|}{Gentoo}                         & \multicolumn{1}{c|}{BSDs}                       & \multicolumn{1}{l|}{OSX}                             \\ \hline
Fedora9 & \multicolumn{1}{c|}{Archlinux}                       & \multicolumn{1}{c|}{OpenSolaris}                 & \multicolumn{1}{c|}{Debian}                       &                                                     &                                                 &                                                      \\ \cline{1-4}
RHEL6   & \multicolumn{1}{c|}{RHEL7}                           & \multicolumn{1}{c|}{illuminos}                   & \multicolumn{1}{c|}{Suse}                         &                                                     &                                                 &                                                      \\ \cline{1-4}
Centos  & \multicolumn{1}{c|}{openSUSE12}                      & \multicolumn{1}{c|}{smartos}                     &                                                   &                                                     &                                                 &                                                      \\ \cline{1-3}
Debian  &                                                      &                                                  &                                                   &                                                     &                                                 &                                                      \\ \cline{1-1}
\end{tabular}
\end{table}

\begin{table}[H]
\centering
\caption{Atributos relacionados a Serviços e daemon}
\label{tab:service}
\begin{tabular}{c|c|c|c|c|c|}
\hline
\rowcolor[HTML]{C0C0C0} 
Atributo                                                           & Origem                                                       & Relevante? & Ohai & Viável & Selecionado? \\ \hline
Status                                                             & \begin{tabular}[c]{@{}c@{}}systemctl/\\ service\end{tabular} & sim        & sim  & sim    & sim          \\ \hline
\begin{tabular}[c]{@{}c@{}}Unidades\\ Rodando\end{tabular}         & \begin{tabular}[c]{@{}c@{}}systemctl/\\ service\end{tabular} & sim        & sim  & sim    & sim          \\ \hline
Falhas                                                             & \begin{tabular}[c]{@{}c@{}}systemctl/\\ service\end{tabular} & sim        & sim  & sim    & sim          \\ \hline
\begin{tabular}[c]{@{}c@{}}Unidades\\ Instaladas\end{tabular}      & \begin{tabular}[c]{@{}c@{}}systemctl/\\ service\end{tabular} & sim        & sim  & sim    & sim          \\ \hline
\begin{tabular}[c]{@{}c@{}}Unidades\\ Status\end{tabular}          & \begin{tabular}[c]{@{}c@{}}systemctl/\\ service\end{tabular} & sim        & sim  & sim    & sim          \\ \hline
\begin{tabular}[c]{@{}c@{}}Relação\\ Unidade\\ Pacote\end{tabular} & \begin{tabular}[c]{@{}c@{}}pacman/\\ dpkg/\\ apt-file/\\ pkgfile\end{tabular} & sim        & não  & sim    & sim          \\ \hline
\end{tabular}
\end{table}

\chapter[Configuração Ruby]{Manipulação de arquivos de configuração em Ruby}
\label{apc:conf-file}

\section{Abordagem do Vagrant}
O Vagrant e o Cupper possuem diversas similaridades, como por exemplo serem ambos
ferramentas de contexto DevOps, possuírem o conceito de ambiente em que estão operando
e dependerem de configurações definidas em um arquivo. Todavia, o Vagrant tem
outro propósito, e prevê configurações de vários ambientes e múltiplas definições e sobrescritas
de configurações para um mesmo ambiente.

A classe mais importante com relação a carregamento de configurações do Vagrantfile
é a classe \textit{Loader} do vagrant.
(\url{https://github.com/mitchellh/vagrant/blob/master/lib/vagrant/config/loader.rb})
Ela possui dois métodos que fazem toda a definição e carregamento das configurações, 
o método \textit{set} e o método \textit{load}. 

Com o método set é possível definir nomes e diversas fontes para o 
carregamento de configurações, sendo que uma instancia de \textit{Loader} 
pode ter configurações já carregadas, e suas fontes cacheadas.

O método \textit{load} carrega as configurações em si, para os diversos ambientes,
e trata da mesclagem de configurações já existentes para ambiente que já foram
configurados. O Vagrant faz uso de \textit{procs} que são objetos que guardam
blocos de código em variáveis locais para fazer o tratamento do código carregado
do Vagrantfile e dos que já foram cacheados.

Uma instancia da classe \textit{Environment} 
(\url{https://github.com/mitchellh/vagrant/blob/master/lib/vagrant/environment.rb})
define os diretórios, acha e instancia o Vagrantfile
(\url{https://github.com/mitchellh/vagrant/blob/master/lib/vagrant/vagrantfile.rb})
, define o \textit{Loader} e passa esses diretórios juntamente com a instancia de \textit{Loader} para o construtor do Cupperfile.

Outra diferença do Cupper e do Vagrant é a de que o vagrant possui a necessidade
de separar as configurações entre as máquinas que ele controla (por ser uma
ferramenta de de gerenciar virtualizações). O Cupper, nas versões recentes, só precisa de
configurações globais referentes a como será a sua extração. Dessa forma para
acessar configurações do ambiente, usando a abordagem do vagrant, é necessário
usar uma instancia de \textit{Environment}, checar pelos Vagrantfiles desse ambiente
e carregar configuração de uma máquina específica definida por esse Vagrantfile.

\section{Abordagem utilizando a Gem \textit{configuration}}
A gem \textit{configuration} possibilita a criação e leitura de arquivos de 
configuração. Sua documentação, e casos de uso podem ser encontrados no link
\url{https://github.com/ahoward/configuration/blob/master/README.rdoc}.

\section{Abordagem mais simples utilizando Ruby puro}
Para desenvolver essa abordagem o artigo \cite{config-file:2011} foi utilizado. Ele
explica tanto como desenvolver com Ruby puro quanto com YAML.

Foi criado um módulo \textit{Config} que segue a abordagem de preparação de
parâmetros explicada no artigo, e o Cupperfile segue a abordagem de atribuição
também explicada no artigo. O módulo \textit{Config} e um exmplo de Cupperfile
podem ser vistos nos Códigos~\ref{code:config} e~\ref{code:cuperfile}


\noindent\begin{minipage}{\textwidth}
  \lstset{style=shell}
  \lstinputlisting[language=Ruby, frame=single, label=code:config, caption=Módulo que carrega as configurações]{editaveis/code/config.rb}
\end{minipage}\hfill

\noindent\begin{minipage}{\textwidth}
  \lstset{style=shell}
  \lstinputlisting[language=Ruby, frame=single, label=code:cuperfile, caption=Exemplo de Cupperfile]{editaveis/code/cupperfile.rb}
\end{minipage}\hfill

\section{Arquivos de configuração}

\noindent\begin{minipage}{\textwidth}
  \lstset{style=shell}
  \lstinputlisting[language=Ruby, frame=single, label=code:cupsconfig, caption=Arquivo de configuração para o CUPS]{editaveis/code/cupsconfig}
\end{minipage}\hfill

\noindent\begin{minipage}{\textwidth}
  \lstset{style=shell}
  \lstinputlisting[language=Ruby, frame=single, label=code:nginxconfig, caption=Arquivo de configuração para o \textit{nginx}]{editaveis/code/nginxconfig}
\end{minipage}\hfill
