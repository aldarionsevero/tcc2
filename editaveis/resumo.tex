\begin{resumo}
 % O resumo deve ressaltar o objetivo, o método, os resultados e as conclusões 
 % do documento. A ordem e a extensão
 % destes itens dependem do tipo de resumo (informativo ou indicativo) e do
 % tratamento que cada item recebe no documento original. O resumo deve ser
 % precedido da referência do documento, com exceção do resumo inserido no
 % próprio documento. (\ldots) As palavras-chave devem figurar logo abaixo do
 % resumo, antecedidas da expressão Palavras-chave:, separadas entre si por
 % ponto e finalizadas também por ponto. O texto pode conter no mínimo 150 e 
 % no máximo 500 palavras, é aconselhável que sejam utilizadas 200 palavras. 
 % E não se separa o texto do resumo em parágrafos.

 % \vspace{\onelineskip}
    
 % \noindent
 % \textbf{Palavras-chaves}: latex. abntex. editoração de texto.

Este trabalho vem apresentar a implementação de melhorias e acréscimo de opções para  apresentação de resultados de buscas em gerenciadores de repositórios de pacotes presentes nas distribuições Linux, com o intuito de apresentar os resultados mais relevantes ao usuário com base nos valores de entrada, simplificar o processo de busca e tratar os casos onde a entrada possuir erros ortográficos, sejam por desconhecimento do nome do pacote, seja por casual escrita incorreta. Tal implementação foi construída a partir de algoritmos de \textit{string matching} presentes na literatura para a ampliação do funcionamento do APT.

 \vspace{\onelineskip}
    
 \noindent
 \textbf{Palavras-chaves}: Linux. Gerenciadores de pacotes. APT. C++. Python. Comunidade Código Aberto.
\end{resumo}
