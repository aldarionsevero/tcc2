\begin{resumo}
  Atualmente há uma grande procura do mercado pela implementação de
  \textit{Development and Operations} (DevOps) em suas organizações como parte da estratégia de vantagem
  competitiva. A cultura DevOps traz consigo conceitos e práticas
  que visam aumentar a quantidade de entrega de \textit{software} em um
  curto período de tempo com maior qualidade e segurança. A
  ferramenta Chef, uma das mais populares entre as ferramentas de
  automação dentro do contexto de DevOps, é um facilitador para
  a orquestração de infraestrutura, aplicando o conceito de
  \textit{Infrastructure as Code}.
  Este trabalho propõe a implementação de uma ferramenta para realizar a 
  engenharia reversa deste conceito, construíndo \textit{scrips} no padrão da 
  ferramenta Chef a partir de um ambiente previamente configurado.

 \vspace{\onelineskip}
 \noindent
    \textbf{Palavras-chaves}: Chef. Infraestrutura como Código. Automação. DevOps.
\end{resumo}
